\documentclass{amsart}
\usepackage[foot]{amsaddr}
\usepackage{bcprules,url,verbatim,multicol,enumerate}

\let\phi=\varphi % default phi looks like empty set
\allowdisplaybreaks
\swapnumbers
\newtheorem{theorem}[subsection]{Theorem}
\newtheorem{lemma}[subsection]{Lemma}
\newtheorem{corollary}[subsection]{Corollary}

\def\MLF!{ML$^\text{F}$}

\usepackage{stmaryrd}
\usepackage[cmtip,all]{xy}
\newcommand{\nc}{\newcommand}
\newcommand{\DotExpr}[2]{#1 #2.\ }
\newenvironment{syntax}{\[\begin{array}{rclr}}{\end{array}\]}

\makeatletter
\renewcommand{\@secnumfont}{\bfseries}
\makeatother
\def\thesubsection{\arabic{section}.\arabic{subsection}}
\def\+={\text{+=}}
\setlength\fboxsep{0pt}

\nc\Abs       {\DotExpr\lambda}
\nc\Add       {\mathrm{add}}
\nc\All       {\DotExpr\forall}
\nc\Allin     {\forall\mathrm{in}}
\nc\Allex     {\forall\mathrm{ex}}
\nc\AbortCC   {\mathrm{abort/cc}}
\nc\ApplyCC   {\mathrm{apply/cc}}
\nc\B         {\mathbb}
\nc\Bool      {\mathrm{Bool}}
\nc\Bot       {\texttt{Bot}}
\nc\BoxIt[1]  {\framebox{\raisebox{0.1ex}[1.5ex][0.2ex]{$#1$}}}
\nc\Brouwer   {\operatorname{Brouwer}}
\nc\Case      {\medbreak\noindent\textbf{Case}~}
\nc\Cap       {\MakeL\cap}
\nc\Closure   {\mathrm{closure}}
\nc\Cps       {\operatorname{cps}}
\nc\Cup       {\MakeL\cup}
\nc\Dn[1]     {(#1 \R \bot) \R \bot}
\nc\Dni       {\mathrm{\neg\neg I}}
\nc\Dom       {\mathrm{dom}}
\nc\Down      {\mathord\downarrow}
\nc\Eco       {\mathit{Eco}}
\nc\Env       {\mathit{Env}}
\nc\env       {\mathit{env}}
\nc\Erase     {\mathrm{erase}}
\nc\Ex        {\DotExpr\exists}
\nc\Exin      {\exists\mathrm{in}}
\nc\Exex      {\exists\mathrm{ex}}
\nc\Exmid     {\text{excluded-middle}}
\nc\False     {\mathrm{false}}
\nc\Fix       {\mathrm{fix}}
\nc\FTV       {\operatorname{FTV}}
\nc\FV        {\operatorname{FV}}
\nc\Given     {\text{ given }}
\nc\Halt      {\mathrm{halt}}
\nc\Id        {\mathit{id}}
\nc\If        {\mathrm{if}}
\nc\Ideals    {\mathcal{I}}
\nc\Indent    {\hspace{3em}}
\nc\Int       {\texttt{Int}}
\nc\Inject    {\mathrm{inject}}
\nc\JCT       {J_{\rm CT}}
\nc\JF        {J_{\rm F}}
\nc\JS        {J_{\rm S}}
\nc\LHS       {\operatorname{LHS}}
\nc\Mgsr      {\operatorname{mgsr}}
\nc\MakeT[1]  {\ \BoxIt{#1}\ }
\nc\MakeL[1]  {\ \mathrm{:\hspace{-2pt}#1\hspace{-2pt}:}\ }
\nc\Or        {\ | \ }
\nc\Pack      {\mathrm{pack}}
\nc\Par       {\subsection}
\nc\Piggy     {\mathrm{Piggy}}
\nc\Plus      {\MakeT+}
\nc\Prune     {\operatorname{prune}}
\nc\Rank      {\mathrm{rank}}
\nc\Recall    {\DotExpr\Xi} % looks like ∃
\nc\Redo      {\operatorname{redo}}
\nc\RHS       {\operatorname{RHS}}
\nc\Sem[1]    {\SemSlim{~#1~}}
\nc\SemSlim[1]{\llbracket#1\rrbracket}
\nc\String    {\texttt{String}}
\nc\Sub       {\sqsubseteq}
\nc\Sup       {\sqsupseteq}
\nc\Tabs      {\DotExpr\Lambda}
\nc\Tech      {\text{technical}}
\nc\Times     {\MakeT\times}
\nc\True      {\mathrm{true}}
\nc\Type      {\mathbf{Type}}
\nc\R         {\rightarrow}
\nc\RL        {\MakeL{\BoxIt\R}}
\nc\RT        {\MakeT\R}
%\nc\Red       {\xymatrix{{}\ar@{~>}[r]&{}}}
\nc\Red       {\rightsquigarrow}
\nc\RR        {\Rightarrow}
\nc\Top       {\texttt{Top}}
\nc\Unpack    {\mathrm{unpack}}
\nc\Undo      {\operatorname{undo}}
\nc\Unify     {\operatorname{unify}}
\nc\Up        {\mathord{\uparrow}}
\nc\Update[1] {#1\Updated}
\nc\Updated   {\text{ updated }}
\nc\VST       {\vdash_{\textrm{ST}}}
\nc\VSub      {\vdash_{\textrm S}}
\nc\Wrong     {\textsc{wrong}}


\def\CT0{$\text{CT}_0$}

\title{Ideal model for pluggable types}


\def\thingsExpressibleInMpsModel{%
subtyping, universal types, union and intersection types,
recursive types, dependent types, and higher kinds%
}


\begin{document}

\begin{samepage}

\maketitle

\newskip\shrunken
\shrunken=-0.5cm plus 20cm minus 20cm

\begin{table}[h]
\caption{Correspondence between syntactic and semantic objects}
\label{corres}
\end{table}
\vskip\shrunken
% ...
\def\miniwidth{0.4\textwidth}
\renewcommand\arraystretch{1.5}
\begin{tabular}{ll}
\hline Syntactic object \hspace{1cm}\null& Semantic object \\


\hline Term & Value \\


\indent\texttt{plus 3 5} & \indent the number $8$ \\

\indent$\Abs x x$ & \indent identity function on $V$\\

\indent$\Abs f (\Abs x f~(x~x))~(\Abs x f~(x~x))$
&\indent
\begin{minipage}[t]{\miniwidth}\raggedright
the function mapping each $f\in V\R V$ to its least
fixed point, and each $v\notin V\R V$ to~$\Wrong$
\end{minipage}
\vspace{1ex}
\\


\hline Type expression & Mathematical object, possibly a type\\


\indent\texttt{Nat} & \indent$\B N_\bot = \{\bot,0,1,2,\ldots\}$ \\

\indent\texttt{Int} & \indent$\B Z_\bot = \{\bot, 0, 1, -1, 2, -2, \ldots\}$\\

\indent$\texttt{Nat}\R\texttt{Int}$ & \indent$\{f\in (V\R V)_\bot \Or f(\B N_\bot)\subseteq \B Z_\bot\}$ \\

\indent$\All\alpha\alpha\R\alpha$ & \indent$\bigcap_{T\in\Type}\{f\in (V\R V)_\bot \Or f(T) \subseteq T\}$ \\

\indent$\alpha\R\beta$ & \indent
\begin{minipage}[t]{\miniwidth}
A certain partial function mapping type environments to types
\end{minipage}
\vspace{1ex}
\\


\hline Judgement & Statement \\


\indent\texttt{plus 3 5} : \texttt{Nat} & \indent$8\in\B N_\bot$ \\

\indent $x:\texttt{Nat}\VST x:\texttt{Nat}$ & \indent
\begin{minipage}[t]{\miniwidth}
If $\rho(x)\in\B N_\bot$, then $\rho(x)\in\B N_\bot$
\end{minipage}\\

\indent
\begin{minipage}[t]{\miniwidth}
$\alpha ; \beta\in[\texttt{Bot},\alpha]\VSub
% $\\\null\hspace{2em}$
\beta\le\beta\R\alpha$
\end{minipage}
&
\indent
\begin{minipage}[t]{\miniwidth}\raggedright
For all $T_\alpha\in\Type$,\\
for all $T_\beta\in\Type$ with
$\{\bot\}\subseteq T_\beta\subseteq T_\alpha$,
we have
$T_\beta\subseteq\{f\in (V\R V)_\bot \Or f(T_\beta)\subseteq\B T_\alpha\}$
(It is a false statement.)
\end{minipage}
\vspace{1ex}
\\


\hline Typing rule & Inference rule \\


\begin{minipage}[t]{\miniwidth}\raggedright
\[\frac{
s:\texttt{Nat}\R\texttt{Int}\qquad
t:\texttt{Nat}
}{s~t:\texttt{Int}}\]
\end{minipage}
&
\indent
\begin{minipage}[t]{\miniwidth}\raggedright
If the term $s$ denotes $f\in\{g\Or g(\B N_\bot)\subseteq \B
Z_\bot\}$ and\\
$t$ denotes $v\in \B N_\bot$, then draw the conclusion that
$s~t$ denotes $f(v)\in Z_\bot$.
\end{minipage}
\vspace{1ex}\\\hline
\end{tabular}

\end{samepage}

\section{Pros and cons}

Bracha (Pluggable type systems) recommends making the type system
of a language optional and independent of the language's runtime
behavior. Doing so confers several benefits.
\begin{enumerate}
\item A useful program can be executed even if it has no type.
\item Type systems can evolve faster than the language itself.
\item Type inference can be made optional as well, so that the
expressiveness of the type system is not bounded by the power of
the inference algorithm.
\end{enumerate}
We demonstrate the technique of denotational pluggable types for
purely functional languages. It achieves all of the above and
more:
\begin{enumerate}\setcounter{enumi}3
\item It is safe for type systems to work together. The composite
of several sound subsystems will continue to reject programs with
runtime type errors.
\item Types defined in different subsystems can interact with
each other. It is possible to call library functions defined
in a different type system.
\end{enumerate}
The technique is applicable to a wide variety of type systems. It
is straightforward to express \thingsExpressibleInMpsModel.

There are some restrictions on what a type can be. For example,
if a program~$f$ has type $T$, then every program $g$ has type
$T$ whenever $g$ is less terminating than $f$ but behaves
otherwise just like $f$. Types cannot separate terminating
programs from nonterminating ones.

The technique guarantees nothing about the type checker's
performance; it may run forever. Writing a nonterminating type
checker is as easy as writing a nonterminating program in Java.

\section{Background}

A runtime type error occurs when a value is used in an unintended
way; examples include adding an integer and a truth value,
dereferencing a non-pointer, and calling a non-function. A type
system is \emph{sound} if no well-typed program encounters
runtime type errors during execution. Soundness is one of the
most important design goals of type systems.

There are several methods to prove a type system sound. We will
discuss two: the syntactic approach, and the domain-theoretic
approach.

\Par{The syntactic approach}

It is the current standard framework for soundness proofs.
\begin{enumerate}
\item Capture the runtime behavior of the language in a
small-step semantics such that terms with runtime type errors are
\emph{stuck}: Neither are they values, nor can they reduce to
other terms.
\item Demonstrate \emph{progress}: A well-typed term is either a
value or reduces to something else.
\item Demonstrate \emph{preservation}: If a term has type $\tau$
then it continues to have type $\tau$ after one reduction.
\end{enumerate}
Together, progress and preservation imply that well-typed terms
never get stuck, and thus cannot experience runtime type errors.

In most circumstances, we can ``append'' to a syntactic soundness
proof to accommodate new runtime behaviors without modifying
existing arguments. But we cannot take two systems proven sound
by the syntactic approach, take the union of their typing rules,
and expect soundness to hold for the result. The two syntactic
soundness proofs gave us progress and preservation for terms
typed purely with rules in one system; they say nothing about
terms typed with a mixture of rules from both systems.

\Par{The domain-theoretic approach}

\begin{enumerate}
\item Capture the runtime behavior of the language in a domain
equation.
\item Define types as certain sets of values in the semantic
domain. Designate a special value $\Wrong$ for runtime type
errors, and make sure it is not a member of any type.
\item Construct an interpretation from terms to values in the
semantic domain. Show that if a term $t$ has type $\tau$, then
$t$ interprets to a member of $\tau$.
\end{enumerate}
Since no type contains $\Wrong$, well-typed programs do not
denote $\Wrong$, and their evaluation may not encounter runtime
type errors.

A domain-theoretic soundness proof is not extensible with new
runtime behaviors. Adding mutation to a purely functional
language, for example, requires a completely new domain equation.
The old arguments have to be rewritten, because their
foundation---the old domain equation---has become obsolete.

However, if we extend the type system without modifying the
runtime behavior, then we can keep interpreting terms into the
old semantic domain. If the new typing rules are sound on their
own, then they already meet the expectation of the old soundness
proof, namely that they assign type $\tau$ only to terms
interpreting to a member of $\tau$. In this way, the old proof
carries over even to terms typed with a mixture of old and new
rules, and type soundness continues to hold.

If types are pluggable, then the language's runtime behavior has
to stay constant in all possible type systems. In this situation,
a domain-theoretic type soundness proof is more extensible than a
syntactic proof.


\section{Roadmap}

To achieve safely pluggable types, we exploit the extensibility
of domain-theoretic type soundness proofs when the runtime system
never changes. These are the steps:
\begin{enumerate}
\item Choose a semantic domain. For purely functional languages,
a domain for untyped lambda calculus suffices.
\item Choose a theory of types for the semantic domain. We employ
the ideal model by MacQueen, Plotkin and Sethi. It can easily
express \thingsExpressibleInMpsModel.
\item Develop type systems such that each typing rule corresponds
to a true statement in the theory of types. This property often
follows from the domain-theoretic soundness proof and incurs no
effort beyond that.
\end{enumerate}
Thus typing rules become lemmas, judgements become statements,
and typing derivations become proofs. Typing terms with a mixture
of type systems is no more than writing proofs with a larger
collection of lemmas.

The next section discusses domains and the MacQueen-Plotkin-Sethi
model of types on a high level. We will only disclose the
technical details absolutely essential for understanding
pluggable types.

The section after that describes how to make type systems
pluggable.

The other sections give examples of pluggable type systems. We
will look at System~F, unboxed impredicative polymorphism,
optional type arguments, and subtyping.


\section{MacQueen-Plotkin-Sethi model of types}

Outline
\begin{enumerate}
\item domain equation
\item reflexive transitive antisymmetric ordering $\Sub$
\item topology, continuous functions, compact elements
\item rank of elements; every compact element has a finite rank
\item types
\item proximity between types
\item contractive, nonexpansive
\item type constructors: finite union, arbitrary intersection,
recursive
\item why pluggable: MPS says typing's undecidable with any
recursively enumerable set of axioms, so we can only build
conservative approximations. There mayn't be a best
approximation, so it'd be good to mix things up.
\end{enumerate}


\section{Hask-like category of types}

Object = type

Morphism = (function value, domain type, range type)

Show it's a category at all ...

... which is not cartesian closed.

Talk about fmap, fold and generic programming.

btw, existence of unique fixed point for contractive functors
means that initial algebras are final coalgebras whose object is
the fixed point and whose morphism is id. which means lists
\emph{are} streams.



\section{How to craft pluggable types}
\label{howto}

We have learnt the notion of types, the nature of function types,
and the construction of recursive types. We will see how to make
type systems pluggable.

Our core language is untyped lambda calculus with constants. Its
standard denotational semantics serves as the interface to the
runtime system modeled by the value domain $V$. Each lambda term
denotes a value in $V$ under a term environment~$\env$, which is
a partial function mapping variables to values in $V$.

\begin{syntax}
t & ::= & &\mbox{untyped lambda term} \\
& & c &\mbox{constant} \\
&|& x &\mbox{variable} \\
&|& \Abs xt &\mbox{abstraction} \\
&|& t~t &\mbox{application}
\end{syntax}
%
\begin{align*}
\Sem c~\env &= v_c & \mbox{designated value for constant }c\\
\Sem x~\env &= \env(x) & \mbox{variable look-up}\\
\Sem{\Abs x t}~\env &=
\rlap{function mapping $v\in V$ to $\Sem t~(\Update\env{x\mapsto v})$}\\
\Sem{t_1~t_2}~\env &=
\begin{cases}
\bot&\rlap{if $\Sem{t_1}~\env=\bot$}\\
f(v)&\rlap{if $\Sem{t_1}~\env=f\in V\R V$ and $\Sem{t_2}~\env=v$}\\
\Wrong&\rlap{if $\Sem{t_1}~\env\notin \{\bot\}\cup(V\R V)$}
\end{cases}
\end{align*}

A pluggable type system may rely on syntax extensions. Each
extended term must ``erase'' to an untyped lambda term, so that
the runtime knows how to execute it. A syntax extension may be an
optional type annotation, such as one on the argument of a lambda
abstraction.
\begin{syntax}
\sigma & ::= & \cdots & \mbox{type expression}\\
t & \+= & \Abs{x:\sigma}t & \mbox{annotated abstraction}\\
\Erase(\Abs{x:\sigma}t) & = & \Abs x \Erase(t) &
\mbox{erasure to core language}
\end{syntax}

Type expressions have no core syntax; they are not even required
to denote a type. For our purpose of ruling out runtime errors,
it suffices that \emph{some} type expressions denote types. A
type expressions of System~F denotes a type only if it is
\emph{closed}, when all its type variables are universally
quantified somewhere. Table~\ref{corres} on page~\pageref{corres}
contains more examples of type expressions.

Pluggable type systems produce \emph{Judgements.} Judgements are
abbreviations of mathematical statements. To treat statements as
concrete data, we can represent them by well-formed formulas of
ZFC (Zermelo-Fraenkel set theory with the axiom of choice). Each
pluggable type system is free to produce its own flavor of
judgements, but all systems should strive to eventually produce
$J_0$-judgements, or typing judgements about closed terms.
\begin{syntax}
J_0
& ::=
& t : \sigma
   & \mbox{typing judgement about a closed term} \\
&& & \mbox{where $t$ is closed and $\sigma$ denotes a type}
\end{syntax}%
The judgement $t:\sigma$ abbreviates ``The value denoted by $t$
under the empty environment is a member of the type denoted by
$\sigma$.'' We say a closed term $t$ is \emph{well-typed} if the
type system in use can produce some $J_0$-judgement $t:\sigma$.
Since $\Wrong$ is not a member of any type, well-typed
expressions do not denote $\Wrong$.

A pluggable type system produces judgements according to a
collection of \emph{typing rules}. In fact, we can think of the
type system as the collection of typing rules. A typing rule is
an inference rule in the style of natural deduction. Each rule
has zero or more antecedents, zero or more side conditions, and
one conclusion. The antecedents are judgements, the side
conditions are statements (i.~e., well-formed formulas of ZFC),
and the conclusion is a judgement. A typing rule corresponds to
the statement ``If all antecedents and side conditions are true,
then the conclusion is true.'' We call a typing rule \emph{sound}
if it corresponds to a provable statement. Sound typing rules are
admissible with respect to the inference rules of natural
deduction (Does this claim need justification?). The typing rule
in the last row of table~\ref{corres} on page~\ref{corres} is
sound.

Judgements are produced by \emph{derivations.} A derivation is a
natural deduction proof: a finite tree built from instances of
typing rules, where the antecedents of every rule instance
coincide with the conclusions of instances immediately above it,
and all side conditions are true. The final conclusion is the
product of the derivation. If the typing rules are sound, then
the product of every derivation has a proof in ZFC. A pluggable
type system is \emph{sound} if all its typing rules are sound. A
sound type system only produces true judgements.

We mix several type systems together by taking union of their
typing rules. The mixture of sound type systems is clearly sound.
If it produces the $J_0$-judgement $t:\sigma$, then we are
certain that $t$ denotes an element of some type, which cannot be
$\Wrong$. In this sense, mixing pluggable type systems preserves
type safety.

We know a term $t$ never raises runtime type errors if some
typing derivation produces a $J_0$-judgement about $t$. It does
not matter how the typing derivation is produced. A practical
implementation of a pluggable type system may be incomplete
without compromising type safety. The type checker may reject
some terms by mistake and loop forever on others, but the terms
it does accept are guaranteed to be type safe. Since the type
system is extensible, writing code with an incomplete type
checker will not create maintenance hell in the future. The user
can start using a type system without waiting for its type
checker to become complete and decidable.



\section{Warm-up example: System~F}

We will phrase System~F in the framework outlined in
\S\ref{howto} so that it can be used together with other type
systems. This particular semantic model of System~F was outlined
in Girard (System~F of variable types: fifteen years later,
\S3.1).

\Par{Syntax extension}

System~F requires annotated abstraction, type abstraction and
type application. The interpretation of any term with these
extensions is identical to the interpretation of its erasure.

\begin{syntax}
t &\+=& &\mbox{syntax extension} \\
&& \Abs{x:\sigma}t &\mbox{annotated abstraction}\\
&|& \Tabs\alpha t &\mbox{type abstraction}\\
&|& t~[\sigma] &\mbox{type application}\\
\end{syntax}%

\begin{align*}
\Erase(\Abs{x:\sigma}t) & = \Abs x\Erase(t) \\
\Erase(\Tabs\alpha t) &= \Erase(t) \\
\Erase(t~[\sigma]) &= \Erase(t)
\end{align*}

\Par{Type expressions}

\begin{syntax}
\sigma &::=& &\mbox{open type expression}\\
&& \iota &\mbox{base type}\\
&|& \alpha &\mbox{type variable}\\
&|& \sigma\R\sigma &\mbox{function type}\\
&|& \All\alpha\sigma &\mbox{universal type}
\end{syntax}%

\Par{Interpretation of open types}

A type environment $\Env$ is a partial function mapping type
variables to types. An open type denotes a partial function
mapping type environments to types.

\begin{align*}
\Sem\iota~\Env &= T_\iota \subseteq B\hspace{2cm}\mbox{designated base type}\\
\Sem\alpha~\Env &= \Env(\alpha)\\
\Sem{\sigma_0\R\sigma_1}~\Env &= (\Sem{\sigma_0}~\Env)\ \RT\ (\Sem{\sigma_1}~\Env)\\
&=\{\bot\}\cup\{f\in V\R V\Or f(\Sem{\sigma_0}~\Env)\subseteq\Sem{\sigma_1}~\Env\}\\
\Sem{\All\alpha\sigma}~\Env &=
\bigcap_{T\in\Type} \Sem\sigma~(\Update\Env{\alpha\mapsto T})
\end{align*}

If an open type expression $\sigma$ has no free type variable,
then we say $\sigma$ is \emph{closed} and denotes the
type~$\Sem\sigma~\emptyset$, where $\emptyset$ is the type
environment defined nowhere.


\Par{Judgements}

\begin{syntax}
\Gamma &::=&& \mbox{typing context}\\
&& \emptyset &\mbox{empty context}\\
&|& x:\sigma,\Gamma &\mbox{term variable binding}\\
\\\\
\Delta &::=&& \mbox{bound type variables}\\
&& \emptyset &\mbox{empty set}\\
&|& \alpha,\Delta &\mbox{type variable binding}
\\\\
\JF &::=
&\Gamma;\Delta\vdash t:\sigma
& \mbox{F-judgement}
\end{syntax}


\Par{Interpretion of F-judgements}

A term environment $\env$ is \emph{compatible} with a typing
context $\Gamma$ under the type environment $\Env$ if for each
$x:\sigma\in\Gamma$ we have $\env(x)\in(\Sem\sigma~\Env)$.

The F-judgement $\Gamma;\Delta\vdash t:\sigma$ abbreviates the
following statement.
\begin{quotation}
For every type environment $\Env$ defined on $\Delta$, for every
term environment $\env$ compatible with $\Gamma$ under $\Env$,
\[
\Sem{t}~\env~\in~\Sem\sigma~\Env.
\]
\end{quotation}~


\Par{Typing rules of Pluggable~F}~

\infrule[T-0]
{\emptyset;\emptyset\vdash t:\sigma\andalso t,\sigma\text{ closed}}
{t:\sigma}

\infrule[T-Con]
{v_c\in \Sem\sigma~\emptyset\andalso\sigma\text{ closed}}
{\Gamma;\Delta\vdash c:\sigma}

\infrule[T-Var]
{x:\sigma\in\Gamma}
{\Gamma;\Delta\vdash x:\sigma}

\infrule[T-Abs]
{x:\sigma_1,\Gamma;\Delta\vdash t:\sigma_2
\andalso x\notin\Gamma}
{\Gamma;\Delta\vdash\Abs{x:\sigma}t : \sigma_1\R\sigma_2}

\infrule[T-App]
{\Gamma;\Delta\vdash t_1:\sigma_2\R\sigma_3
\andalso\Gamma;\Delta\vdash t_2:\sigma_2}
{\Gamma;\Delta\vdash t_1~t_2:\sigma_3}

\infrule[T-Tabs]
{\Gamma;\alpha,\Delta\vdash t:\sigma\andalso\alpha\notin\Delta}
{\Gamma;\Delta\vdash\Tabs\alpha t:\All\alpha\sigma}

\infrule[T-Tapp]
{\Gamma;\Delta\vdash t:\All\alpha\sigma_0}
{\Gamma;\Delta\vdash t~[\sigma_1]:\sigma_0[\alpha\mapsto\sigma_1]}


\Par{Soundness of Pluggable~F}

We outline the proof of each statement corresponding to a typing
rule of Pluggable~F.

\textsc{T-0}: Since $t$ and $\sigma$ are closed, they denote a
value $v_t$ and a type $T_\sigma$ under whatever environments.
The antecedent says $v_t\in T_\sigma$; the conclusion says the
same thing.

\textsc{T-Con}: Since $c$ and $\sigma$ are both closed, their
denotations do not depend on type or term environments. The
conclusion is a consequence of the first side condition.

\textsc{T-Var}: Let $\Env$ be an arbitrary type environment
defined on $\Delta$. The side condition implies that no matter
which term environment $\env$ compatible with $\Gamma$ is chosen,
the value $\env(x)$ must be a member of the type
$\Sem\sigma~\Env$. The conclusion follows.

\textsc{T-Abs}: Let $\Env$ be an arbitrary type environment
defined on $\Delta$. By the antecedent, $t$ denotes a value in
$\Sem{\sigma_2}~\Env$ under every environment $\env$ that is
compatible with $\Gamma$ and maps $x$ to a member of
$\Sem{\sigma_1}~\Env$. It follows that the denotation of $\Abs
xt$ under every environment compatible with $\Gamma$ is a
function whose image of $\Sem{\sigma_1}~\Env$ is a subset of
$\Sem{\sigma_2}~\Env$.

\textsc{T-App}: Let $\Env$ be an arbitrary type environment
defined on $\Delta$. The antecedents say that under every
compatible environment, $t_1$ denotes a function $f$ whose image
of $\Sem{\sigma_2}~\Env$ is a subset of $\Sem{\sigma_3}~\Env$,
and $t_2$ denotes a value $v\in\Sem{\sigma_2}~\Env$. As desired,
$f(v)\in\Sem{\sigma_3}~\Env$.

\textsc{T-Tabs}: Let $\Env$ be an arbitrary type environment
defined on $\Delta$. Let $\env$ be an arbitrary term environment
compatible with $\Gamma$ under $\Env$. Let $v=\Sem{t}~\env$.
For every type $T\in\Type$, the antecedent guarantees that
\[
v\in\Sem\sigma~(\Update\Env(\alpha\mapsto T)),
\]
which implies
\[
v\in\bigcap_{T\in\Type}\Sem\sigma~(\Update\Env(\alpha\mapsto T))
=\Sem{\All\alpha\sigma}~\Env.
\]

\textsc{T-Tapp}: Let $\Env$ be an arbitrary type environment
defined on $\Delta$. Let $\env$ be an arbitrary term environment
compatible with $\Gamma$ under $\Env$. Write
\begin{align*}
T_1&=\Sem{\sigma_1}~\Env,\\
v&=\Sem{t}~\env.
\end{align*}
By the antecedent,
\[
v\in
\bigcap_{T\in\Type}\Sem{\sigma_0}~(\Update\Env(\alpha\mapsto T))
\ \subseteq\ \Sem{\sigma_0}~(\Update\Env\alpha\mapsto T_1).
\]
By the correctness of capture-avoiding substitution,
\[
\Sem{\sigma_0}~(\Update\Env\alpha\mapsto T_1)
=
\Sem{\sigma_0[\alpha\mapsto\sigma_1]}~\Env,
\]
which gives us $v\in\Sem{\sigma_0[\alpha\mapsto\sigma_1]}~\Env$
as desired.


\section{Constrained type system version 1}



\section{Mixing F and CT1 for optional type arguments}



\section{Parametric types}

It may be possible to refine ``equal up to termination'' relation
in ``fast and loose reasoning'' to the non-transitive consistency
relation. Free theorems should hold. Parametric types are not
universal types.


\begin{comment}

\appendix
% nitty gritty details

\section{What is a type?}

MacQueen, Plotkin and Sethi proposed a notion of types as sets in
a semantic domain of untyped lambda calculus. These types are
rough descriptions of the behavior of their inhabitants. For
example, if a function $f$ has type $\mathbb Z\R\mathbb Z$, then
$f$ is guaranteed to map an integer to another integer. If $g$
has type $\All\alpha \alpha\R\alpha$, then $g(x)$ inhabits every
type where $x$ is an inhabitant. Under this scheme, a recursive
type equation such as
\[
\mathit{List} = \{\mathit{nil}\} + (\mathbb Z \times \mathit{List})
\]
has a unique solution, and thereby defines a type.

In the MacQueen-Plotkin-Sethi model, a type is a nonempty
Scott-closed subset of the value domain $V$. The existence and
uniqueness proof for solutions of recursive type equations
depends crucially on the properties of Scott-closed sets.


\Par{Value domain}

$V$ is a set containing all functions definable in untyped lambda
calculus. It is the solution of the following equation up to
set-theoretic isomorphism:
\begin{equation}\label{domain-eq}
V \cong B + (V + V) + (V \times V) + (V \R V) + \{\bot, \top\}.
\end{equation}
Here, $B$ is any set of base values. In a practical programming
language, its members are often truth values, integers and
floating point numbers. The symbol $+$ denotes disjoint union,
and $\times$ denotes cartesian product. They give rise to a
\emph{approximation} partial order $\Sub$ under which $\bot$ is
always the least element, $\top$ is always the greatest element,
and members of $B$ are incomparable with one another. The
approximation partial order $\Sub$ determines a Scott topology on
$V$, and the function space $(V \R V)$ consists of the continuous
functions of the Scott topology. We will talk about Scott
topologies later. For now, it suffices to appreciate that $V$
contains base values, function values and values of (not
necessarily positive) algebraic data types.

There are a number of ways to construct a set $V$ satisfying
equation~\eqref{domain-eq}. We use the construction producing a
\emph{consistently complete algebraic cpo}. We will use the
property of $V$ as a cpo in the definition of types, and use the
algebraicity of $V$ to establish that recursive algebraic data
types are well-defined. Consistent completeness is unused in our
development.

Previous works define an extra member $\Wrong$ of $V$ to stand
for runtime type errors. We use the maximum element $\top$ for
runtime type errors instead, so that $\Wrong$ does not have to be
purposefully excluded from every type. In fact, our types are
closed nonempty proper subsets of $V$ under its Scott topology.

\Par{Complete partial orders}

Let $\sqsubseteq$ be a partial order over $V$. A subset of $V$ is
\emph{directed} if every two members $x$, $y$ of $A$ has an upper
bound $z$ in $A$ such that $z\sqsupseteq x$ and $z\sqsupseteq y$.
The partially ordered set $V$ is a \emph{complete partial order},
or \emph{cpo}, if every directed subset of $V$ has a supremum
(i.~e., least upper bound) in $V$.

Our chosen solution of equation~\eqref{domain-eq} is a cpo. We
will henceforth use that fact without further qualification.

\Par{Scott topology}

The solution to domain equation~\eqref{domain-eq} should impose a
\emph{approximation} partial order $\Sub$ on $V$ satisfying the
following properties.
\begin{itemize}
\item $\bot$ is the least element: $\bot\Sub v$ for all $v\in V$.
\item $\top$ is the greatest element: $v\Sub\top$ for all $v\in
V$.
\item Base values are incomparable: If $u,v\in B$, then
$u\not\Sub v$ and $v\not\Sub u$.
\end{itemize}
A topology of $V$ is any labeling of subsets of $V$ as
\emph{open} such that
\begin{itemize}
\item $\emptyset$ and $V$ are open,
\item the union of any family of open sets is open,
\item the intersection of a finite number of open sets is open.
\end{itemize}
We mentioned that the function space $(V\R V)$ consists of the
continuous functions according to the Scott topology. A function
is continuous if its preimage of each open set is an open set.

\begin{samepage}
The Scott topology of $V$ labels a subset $S$ open if it
satisfies the following conditions.
\begin{itemize}
\item $S$ is \emph{upward-closed}: If $u\in S$ and $u\Sub v$,
then $v\in S$.
\item $S$ is \emph{inaccessible by directed supremums}: If $A$ is
a directed set disjoint from $S$, then the supremum of $A$ is
outside $S$.
\end{itemize}
\end{samepage}

A set is \emph{closed} if it is the complement of an open set.

\begin{lemma}[Characterizing closed sets of Scott topology]
A set $S\subseteq V$ is closed under Scott topology if and
only if it satisfies the following conditions.
\begin{itemize}
\item $S$ is \emph{downward-closed}: If $v\Sub u$ and $u\in S$,
then $v\in S$.
\item $S$ is \emph{closed under directed supremums}: If
$A$ is a directed subset of $S$, then the supremum of $A$ is a
member of $S$.
\end{itemize}
\end{lemma}

\begin{proof}
$S$ is downward-closed if and only if its complement is
upward-closed. $S$ is closed under directed supremums if and only
if its complement is inaccessible by directed supremums.
\end{proof}

\Par{Types}

A subset of $V$ is a \emph{type} if it is a proper closed set
under the Scott topology. In other words, types are nonempty
proper subsets of $V$ closed downward and under directed
supremums. If $v\in T$, then we say that the value $v$
\emph{inhabits} the type $T$, and that $v$ is an
\emph{inhabitant} of $T$.

We will use the property of types as closed sets to establish
well-approximation of recursive algebraic data types.

\begin{lemma}
$\bot$ inhabits every type. $\top$ does not inhabit any type.
\end{lemma}

\section{Recursively defined types}

\Par{Road map}

Let us reproduce the result by MacQueen, Plotkin and Sethi that
types can be defined by recursive equations. For example, the
type of heterogeneous lists is defined by the equation
\[
\texttt{List} = \texttt{Unit} + (V \times \texttt{List}),
\]
which holds with set-theoretic identity. The argument has 3
steps.
\begin{enumerate}[(i)]
\item Define the distance between every pair of types, and
establish completeness of the resulting metric space in the sense
that every Cauchy sequence of types converges to a type.
\item Show that the product, sum and function type constructions
are contractive in both arguments,
\item Invoke the Banach fixed-point theorem on the nonempty
complete metric space of types to show that every contraction
mapping between types has a unique fixed point. By (2), a
recursive type equation defines a unique type if its right hand
side uses only the product, sum and function type constructions.
\end{enumerate}

Step~(i) works for every distance function in a certain family.
But step~(ii) works for only one particular distance function,
whose definition depends crucially on the properties of $V$ as an
algebraic cpo.

\Par{Compact elements and algebraic cpos}

An element $v\in V$ is \emph{compact} if for all directed set $A$
such that $v\Sub\sup A$, there exists $u\in A$ such that $v\Sub
u$.

For any subset $S\subseteq V$, write $S_c$ for the set of compact
elements in $S$. A cpo $V$ is \emph{algebraic} if every $v\in V$
is the supremum of smaller compact elements:
\[
v=\sup \{u \in V_c \Or u \Sub v \}.
\]
Our chosen solution of equation~\eqref{domain-eq} is an algebraic
cpo. We will use its algebraicity without ceremony from now on.

\begin{lemma}
Let $V$ be an algebraic cpo. A subset $S$ contains every compact
element, or $S\supseteq V_c$, if and only if for every $v\in V$,
\[
v=\sup\{u\in S \Or u\Sub v\}.
\]
\end{lemma}

\begin{proof}
($\Rightarrow$) By algebraicity and because $V_c\subseteq S$,
\[
v=\sup\{u\in V_c \Or u \Sub v\} \Sub \sup\{u\in S \Or u \Sub v\}.
\]
Since $v$ is an upper bound of $\{u\in S \Or u \Sub v\}$, we also
have
\[
v\Sup \sup\{u\in S \Or u \Sub v\}.
\]
The proof goal follows from antisymmetry of the approximation
partial order $\Sub$.

($\Leftarrow$) Choose arbitrary $v\in V_c$. We are to show $v\in
S$. Let $A= \{u \in S \Or u \Sub v\}$. By assumption $v=\sup A$.
Since $v$ is compact, there exists $u'\in A$ such that $v\Sub
u'$. But we also have $u'\Sub v$ by definition of $A$, which
implies $v=u'\in A\subseteq S$.
\end{proof}

\Par{Rank}

\begin{lemma}
Every compact element has a rank.
\end{lemma}

\Par{Converging sequences of sets}
Let us recall the standard definition of convergence.

The \emph{limit superior} of an infinite sequence of sets
$S_1,S_2,\ldots$ is
\[
\limsup_{n\rightarrow\infty}S_n =
\bigcap_{n=1}^\infty\bigcup_{i = n}^\infty S_i.
\]
The \emph{limit inferior} is
\[
\liminf_{n\rightarrow\infty}S_n =
\bigcup_{n=1}^\infty\bigcap_{i = n}^\infty S_i.
\]
If the limit superior and limit inferior are equal, then the
sequence $S_1,S_2,\ldots$ \emph{converges}, and its \emph{limit}
is
\[
S = \limsup_{n\rightarrow\infty}S_n = \liminf_{n\rightarrow\infty}S_n.
\]

\Par{Metric space of types}

Let ``$\Rank$'' be an arbitrary function from $V$ to $\mathbb R$.
The proximity of two types $S$, $T$ is the smallest rank of a
value in the symmetric difference of $S$ and $T$. If $S=T$, then
their proximity is $\infty$. The distance $d(S, T)$ between two
types is inverse exponential in their proximity:
\[
d(S,T) =
\left(\frac12\right)^{\mathrm{proximity}(S, T)}
%\frac1{2 ^{\mathrm{proximity}(S, T)}} % looks worse
 \]
It is easy to verify that the distance function $d$ satisfies the
requirements for forming a metric space over the set of types:
It is nonnegative, evaluates to $0$ only on equal types, is
commutative, and satisfies a stronger inequality than the
triangle inequality:
\[
d(R,T)\le\max(d(R,S),d(S,T)).
\]
To see it, let $v$ be the value of minimum rank $r$ in the
symmetric difference between $R$~and $T$, and suppose $v\in R-T$.
Then $d(R, T)=2^{-r}$. If $v\in S$, then $v\in S-T$ and
$d(S,T)\ge2^{-r}$. If $v\notin S$, then $d(R, S)\ge2^{-r}$.

\Par{Cauchy sequences}

An infinite sequence $T_1,T_2,\ldots$ of types is Cauchy if for
every $\epsilon > 0$ there exists $n\in\mathbb N$ such that for
all $i,j\ge n$, the distance between $T_i$~and $T_j$ is smaller
than $\epsilon$.

\begin{theorem}
No matter which rank function is chosen, the metric space of
types is complete in the sense that every Cauchy sequence of
types converges to a type.
\end{theorem}

\begin{proof}
There are two proof goals.
\begin{enumerate}
\item Every Cauchy sequence of types converges.
\item The limit of a Cauchy sequence of types is a type.
\end{enumerate}

Part (1). Let $T_1,T_2,\ldots$ be a Cauchy sequence of types.
Knowing the standard result $\liminf T_n\subseteq\limsup T_n$,
we need only show $\limsup T_n\subseteq\liminf T_n$.

\def\Cauchy{\texttt{cauchy}}
\def\Hasv{{\texttt{has\textunderscore}v}}

Choose an arbitrary value
\[
v\in \limsup T_n = \bigcap_{n=1}^\infty\bigcup_{i = n}^\infty T_i.
\]
Let $\epsilon=2^{-\Rank(v)}>0$. There exists a natural number
$\Cauchy$ such that for all $i,j\ge\Cauchy$, we have
$d(T_i,T_j)<\epsilon$. Since $v$ is a member of the limit
superior, there exists a natural number $\Hasv\ge\Cauchy$ such
that $v\in T_\Hasv$. If we pick any $i\ge\Hasv$, then we have
$v\in T_i$, because otherwise the contradiction
\[
d(T_\Hasv, T_i)\ge2^{-\Rank(v)}=\epsilon
\]
would arise. Thus
\[
v\in
\bigcap_{i=\Hasv}^\infty T_i\subseteq
\bigcup_{n=1}^\infty\bigcap_{i = n}^\infty T_i.
=\liminf T_n.
\]

Part (2). We will verify that the limit
\[
T=\limsup T_n=\liminf T_n
\]
is a nonempty proper subset of $V$ closed downward and closed
under directed supremum.
\end{proof}

\end{comment}

\end{document}
