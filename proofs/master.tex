\documentclass{amsart}
\usepackage{bcprules,url,enumerate}
\allowdisplaybreaks
\swapnumbers
\newtheorem{theorem}[subsection]{Theorem}
\newtheorem{lemma}[subsection]{Lemma}
\newtheorem{corollary}[subsection]{Corollary}
\theoremstyle{definition}
\newtheorem{definition}[subsection]{Definition}

\def\thesubsection{\arabic{subsection}}

\title{Nuclear Football Megaslave}

\def\MLF!{ML$^\text{F}$}

\usepackage{stmaryrd}
\usepackage[cmtip,all]{xy}
\newcommand{\nc}{\newcommand}
\newcommand{\DotExpr}[2]{#1 #2.\ }
\newenvironment{syntax}{\[\begin{array}{rclr}}{\end{array}\]}

\makeatletter
\renewcommand{\@secnumfont}{\bfseries}
\makeatother
\def\thesubsection{\arabic{section}.\arabic{subsection}}
\def\+={\text{+=}}
\setlength\fboxsep{0pt}

\nc\Abs       {\DotExpr\lambda}
\nc\Add       {\mathrm{add}}
\nc\All       {\DotExpr\forall}
\nc\Allin     {\forall\mathrm{in}}
\nc\Allex     {\forall\mathrm{ex}}
\nc\AbortCC   {\mathrm{abort/cc}}
\nc\ApplyCC   {\mathrm{apply/cc}}
\nc\B         {\mathbb}
\nc\Bool      {\mathrm{Bool}}
\nc\Bot       {\texttt{Bot}}
\nc\BoxIt[1]  {\framebox{\raisebox{0.1ex}[1.5ex][0.2ex]{$#1$}}}
\nc\Brouwer   {\operatorname{Brouwer}}
\nc\Case      {\medbreak\noindent\textbf{Case}~}
\nc\Cap       {\MakeL\cap}
\nc\Closure   {\mathrm{closure}}
\nc\Cps       {\operatorname{cps}}
\nc\Cup       {\MakeL\cup}
\nc\Dn[1]     {(#1 \R \bot) \R \bot}
\nc\Dni       {\mathrm{\neg\neg I}}
\nc\Dom       {\mathrm{dom}}
\nc\Down      {\mathord\downarrow}
\nc\Eco       {\mathit{Eco}}
\nc\Env       {\mathit{Env}}
\nc\env       {\mathit{env}}
\nc\Erase     {\mathrm{erase}}
\nc\Ex        {\DotExpr\exists}
\nc\Exin      {\exists\mathrm{in}}
\nc\Exex      {\exists\mathrm{ex}}
\nc\Exmid     {\text{excluded-middle}}
\nc\False     {\mathrm{false}}
\nc\Fix       {\mathrm{fix}}
\nc\FTV       {\operatorname{FTV}}
\nc\FV        {\operatorname{FV}}
\nc\Given     {\text{ given }}
\nc\Halt      {\mathrm{halt}}
\nc\Id        {\mathit{id}}
\nc\If        {\mathrm{if}}
\nc\Ideals    {\mathcal{I}}
\nc\Indent    {\hspace{3em}}
\nc\Int       {\texttt{Int}}
\nc\Inject    {\mathrm{inject}}
\nc\JCT       {J_{\rm CT}}
\nc\JF        {J_{\rm F}}
\nc\JS        {J_{\rm S}}
\nc\LHS       {\operatorname{LHS}}
\nc\Mgsr      {\operatorname{mgsr}}
\nc\MakeT[1]  {\ \BoxIt{#1}\ }
\nc\MakeL[1]  {\ \mathrm{:\hspace{-2pt}#1\hspace{-2pt}:}\ }
\nc\Or        {\ | \ }
\nc\Pack      {\mathrm{pack}}
\nc\Par       {\subsection}
\nc\Piggy     {\mathrm{Piggy}}
\nc\Plus      {\MakeT+}
\nc\Prune     {\operatorname{prune}}
\nc\Rank      {\mathrm{rank}}
\nc\Recall    {\DotExpr\Xi} % looks like ∃
\nc\Redo      {\operatorname{redo}}
\nc\RHS       {\operatorname{RHS}}
\nc\Sem[1]    {\SemSlim{~#1~}}
\nc\SemSlim[1]{\llbracket#1\rrbracket}
\nc\String    {\texttt{String}}
\nc\Sub       {\sqsubseteq}
\nc\Sup       {\sqsupseteq}
\nc\Tabs      {\DotExpr\Lambda}
\nc\Tech      {\text{technical}}
\nc\Times     {\MakeT\times}
\nc\True      {\mathrm{true}}
\nc\Type      {\mathbf{Type}}
\nc\R         {\rightarrow}
\nc\RL        {\MakeL{\BoxIt\R}}
\nc\RT        {\MakeT\R}
%\nc\Red       {\xymatrix{{}\ar@{~>}[r]&{}}}
\nc\Red       {\rightsquigarrow}
\nc\RR        {\Rightarrow}
\nc\Top       {\texttt{Top}}
\nc\Unpack    {\mathrm{unpack}}
\nc\Undo      {\operatorname{undo}}
\nc\Unify     {\operatorname{unify}}
\nc\Up        {\mathord{\uparrow}}
\nc\Update[1] {#1\Updated}
\nc\Updated   {\text{ updated }}
\nc\VST       {\vdash_{\textrm{ST}}}
\nc\VSub      {\vdash_{\textrm S}}
\nc\Wrong     {\textsc{wrong}}

\begin{document}

\maketitle
\tableofcontents

\subsection{Syntax}
\label{syntax}
\begin{syntax}
\mbox{type}\\
\sigma, \tau
&::=& \iota &\mbox{base type}\\
&|& \alpha & \mbox{type variable} \\
&|& \tau \rightarrow \tau & \mbox{function type} \\
&|& \tau\R\tau\Given C&\mbox{constrained function type}\\
&|& \All\alpha\tau&\mbox{universal type, as long as $\tau$ is}\\
&&&\quad\mbox{neither base type nor type variable}
\\
\\
\mbox{constraints}\\
C
&::=&\emptyset\\
&|& \tau\Sub\tau,C
\\
\\
\mbox{term}\\
s, t & ::= & x & \mbox{variable} \\
& | & \Abs x t  & \mbox{lambda abstraction} \\
& | & t_1~t_2 & \mbox{function application} \\
& | & c~t_1~\cdots~t_n& \mbox{$n$-nary constant}
\end{syntax}

\subsection{Subtyping}

\begin{syntax}
\mbox{list of type variables}\\
A,E
&::=& \emptyset \\
&|& \alpha,A
\\
\\
\mbox{list of constraints}\\
C
&::=& \emptyset \\
&|& \sigma \Sub \tau, C
\\
\\
\mbox{subtype judgement}
&::=&A;E\vdash C
\end{syntax}

\infrule[S-vacuous]
{}
{A;E\vdash\emptyset}

\infrule[S-Refl]
{\FTV(\tau)\subseteq A\cup E
\andalso
A;E\vdash C
}
{A;E \vdash \tau\Sub\tau,C}

% Reason of existence: sublem
\infrule[S-Dupe]
{A;E\vdash \sigma\Sub\tau, C}
{A;E\vdash \sigma\Sub\tau, \sigma\Sub\tau, C}

% Reason of existence: sublem.
\infrule[S-Trans]
{A;E\vdash \tau_0\Sub\tau_1,\tau_1\Sub\tau_2,C}
{A;E\vdash \tau_0\Sub\tau_2,C}

\infrule[S-Arrow]
{A;E\vdash\tau_0\Sub\sigma_0,\sigma_1\Sub\tau_1,C}
{A;E\vdash\sigma_0\R\sigma_1\Sub\tau_0\R\tau_1,C}

\infrule[S-Loner]
{\alpha\notin A\cup E\cup\FTV(C)
\andalso
A;E\vdash
\{ \sigma_i\Sub\tau_j \Or 1 \le i \le m, 1 \le j \le n \}
\cup C
}{
\alpha,A;E \vdash
\sigma_1\Sub\alpha,\ldots,\sigma_m\Sub\alpha,
\alpha\Sub\tau_1,\ldots,\alpha\Sub\tau_n,C
}

\infrule[S-LI]
{\alpha\notin A\cup E\cup\FTV(\tau)\cup\FTV(C)
\andalso
\alpha,A;E\vdash\sigma\Sub\tau,C}
{A;E\vdash(\All\alpha\sigma)\Sub\tau,C}

\infrule[S-LE]
{\alpha\notin A\cup E\cup\FTV(\tau)\cup\FTV(C)
\andalso
A;E\vdash(\All\alpha\sigma)\Sub\tau,C}
{\alpha,A;E\vdash\sigma\Sub\tau,C}

\infrule[S-RI]
{\alpha\notin A\cup E\cup\FTV(\tau)\cup\FTV(C)
\andalso
A;\alpha,E\vdash\sigma\Sub\tau,C}
{A;E\vdash\sigma\Sub(\All\alpha\tau),C}

\infrule[S-CI]
{A;E\vdash \sigma\Sub\tau,C_\sigma\cup C}
{A;E\vdash (\sigma\Given C_\sigma)\Sub\tau,C}

\infrule[S-CE]
{A;E\vdash (\sigma\Given C_\sigma)\Sub\tau,C}
{A;E\vdash \sigma\Sub\tau,C_\sigma\cup C}

Note that there are no analogues of \textsc{S-CI} or
\textsc{S-CE} about constrained types on right hand side of a
subtype constraint. There is the introduction rule \textsc{S-RI}
without a corresponding elimination rule.

\subsection{Constrained typing}
\begin{syntax}
\mbox{constrained typing judgement}
&::=&\Gamma \vdash t : \tau \Given C
\end{syntax}%
\infrule[CT-Var]
{x:\tau\in\Gamma}
{\Gamma\vdash x : \tau \Given \emptyset}

\infrule[CT-Abs]
{
\{\alpha_1,\ldots,\alpha_n\}
\subseteq
\FTV(\sigma,\tau,C)-\FTV(\Gamma)
\andalso
\Gamma,x:\sigma \vdash t : \tau \Given C}
{\Gamma\vdash (\Abs x t) :
(\All{\alpha_1\cdots\alpha_n}\sigma\R\tau\Given C) \Given C}

\infrule[CT-App]
{
\Gamma\vdash s : \sigma \Given C_1
\andalso
\Gamma\vdash t : \tau \Given C_2}
{\Gamma\vdash s~t : \beta \Given
\sigma\Sub\alpha\R\beta,\tau\Sub\alpha,C_1\cup C_2}

\infrule[CT-Con]
{
\Gamma\vdash s_1:\sigma_1\Given C_1
\andalso
\cdots
\andalso
\Gamma\vdash s_n:\sigma_n\Given C_n
}
{\Gamma\vdash c~s_1~\cdots~s_n:\tau_c\Given
\sigma_1\Sub\tau_1,\ldots,\sigma_n\Sub\tau_n,
{\textstyle\bigcup_{i=1}^n C_i}
}

In \textsc{CT-Con}, the types $\tau_c,\tau_1,\ldots,\tau_n$
depends on $c$ alone. One constant $c$ may have multiple
\textsc{CT-Con} rules, as long as the corresponding
$\delta$-typability requirement is satisfied
(\S\ref{eval}). Here are some examples.

\infrule[CT-Add]
{\Gamma\vdash s_1:\sigma_1\Given C_1
\andalso
\Gamma\vdash s_2:\sigma_2\Given C_2
}
{\Gamma\vdash\Add~s_1~s_2:
\Int\Given\sigma_1\Sub\Int,\sigma_2\Sub\Int,C_1\cup C_2}

\infrule[CT-If]
{\Gamma\vdash s:\sigma\Given C}
{\Gamma\vdash\If~s:
(\All\alpha\alpha\R\alpha\R\alpha)\Given\sigma\Sub\Bool,C}

\infrule[CT-Fix]
{\Gamma\vdash s:\sigma\Given C}
{\Gamma\vdash\Fix~s:\alpha\Given\sigma\Sub \alpha\R\alpha,C}

\subsection{Evaluation}
\label{eval}

\begin{syntax}
\mbox{value}\\
v&::=&c&\mbox{nullary constant of base type}\\
&|&\Abs{x:\sigma}t&\mbox{lambda abstraction}
\end{syntax}

\infrule[$\beta$]
{}
{(\Abs{x:\sigma}s)~t \Red s[x\mapsto t]}

\infrule[$\delta$]
{}
{c~v_1~\cdots~v_n \Red v'}

\infrule[E-App]
{s\Red s'}
{s~t\Red s'~t}

\infrule[E-Con]
{t_i\Red t_i'}
{c~t_1~\cdots~t_i~\cdots~t_n
\Red
c~t_1~\cdots~t_i'~\cdots~t_n
}

The congruence rule \textsc{E-App} implements call-by-name so
that we write the hardest proof. With call-by-value, progress is
easier and preservation is no harder.

The congruence rule \text{E-Con} says that primitive operators
are supposed to be strict, but they are free to force their
arguments in any order.

Rule \textsc{$\delta$} represents all $\delta$-reductions. The
$\delta$-reductions are required to satisfy what
Wright~and Felleisen~\cite{Wright94} call $\delta$-typability:
\begin{align*}
&\text{Let $c$ be the constant in \textsc{CT-Con}.}\\[-4pt]
\tag{$\delta$-typeability, informal version}\label{delta-typability}
&\text{For each value $v_i$ with type $\tau_i$,}\\[-4pt]
&\text{there exists a value $v'$ of type $\tau_c$ such that}\\
&\hspace{2em}c~v_1~\cdots~v_n\Red v'.
\end{align*}

By ``nullary constant of base type'', we mean any nullary
constant $c$ for which there is a constrained typing rule
\[
\Gamma\vdash c:\iota\Given\emptyset.
\]
Nullary constants of function type, if exist, are not values. By
$\delta$-typability, each of them evaluates to a lambda
abstraction.

Here are some $\delta$-rules examples.
\begin{align*}
\Add~1~1&\Red2\\
\Add~1~2&\Red3\\
\Add~38701589~8472584&\Red47174173
\end{align*}
One can imagine $\delta$-rules such as
\begin{align*}
\If~\True &\Red\Abs{x:\alpha}\Abs{y:\alpha}x,\\
\If~\False&\Red\Abs{x:\alpha}\Abs{y:\alpha}y,\\
\Fix~(\Abs{x:\sigma} t)&\Red t[x\mapsto \Fix~(\Abs{x:\sigma} t)].
\end{align*}
Proving them to satisfy $\delta$-typability is nontrivial,
though.

\subsection{Substitution lemma}

We set out to prove that if $t$ can be typed $\tau$, then after
a well-typed substitution it can still be typed $\tau$.

\def\Exi{\emptyset}
\def\s{s}
\def\sigx{\sigma_x}
\def\sigs{\sigma_\s}
\def\sigt{\sigma_t}
\def\sigts{\sigma_t^\s}
\def\Ct{C_t}
\def\Cs{C_\s}
\def\Cts{C_t^\s}

Suppose the following is derivable for some $C\supseteq \Ct\cup \Cs$.
\begin{align}
\Gamma,x:\sigx&\vdash t:\sigt\Given \Ct \notag
\\
\Gamma&\vdash \s:\sigs \Given \Cs \notag
\\
A;\Exi&\vdash
\sigs\Sub\sigx, \sigt\Sub\tau,C
\label{eq:subass}
\end{align}
Then under the aforementioned typing context $\Gamma$, we can
derive
\begin{align*}
\Gamma&\vdash t[x\mapsto\s]:\sigts \Given \Cts\\
A';\Exi&\vdash \sigts\Sub\tau, C'
\end{align*}
for some $C'\supseteq \Cts \cup C$.

\begin{proof}
By induction on the structure of $t$.

\Case$t=y$, $y\neq x$. The assumption satisfies the proof goal.

\Case$t=x$. Set $\sigts=\sigs$ and $\Cts=\Cs$. The presumed
derivation of
\[
\Gamma\vdash \s : \sigs \Given \Cs
\]
is the desired constrained typing derivation. From the presumed
derivation of
\[
A;\Exi\vdash \sigs\Sub\sigx, \sigx\Sub\tau,C,
\]
we derive the desired subtype judgement via one application of
\textsc{S-Trans}:
\[
A;\Exi\vdash \sigs\Sub\tau,C.
\]

\Case$t=\Abs y t_0$. Let this be the last step in the presumed
constrained typing derivation of $t$:

\def\sigay{\sigma_y^\alpha}
\def\siga0{\sigma_0^\alpha}
\def\Ca0{C_0^\alpha}

\def\sigby{\sigma_y^\beta}
\def\sigb0{\sigma_0^\beta}
\def\Cb0{C_0^\beta}


\infrule[CT-Abs]
{\{\alpha_1,\ldots,\alpha_n\}\subseteq
\FTV(\sigay,\siga0,\Ca0) - \FTV(\Gamma,\sigx)
\andalso
\Gamma,x:\sigx,y:\sigay\vdash t_0 : \siga0\Given\Ca0
}
{\Gamma,x:\sigx\vdash(\Abs y t_0):
(\All{\alpha_1\cdots\alpha_n}\sigay\R\siga0 \Given \Ca0)
\Given\Ca0}

Choose fresh type variables $\beta_1,\ldots,\beta_n$. Since
$\alpha_1,\ldots,\alpha_n$ are disjoint from $\Gamma$ and
$\sigx$, we can rename them uniformly in the constrained typing
derivation of $t_0$ to obtain
\begin{align}
\Gamma,x:\sigx,y:\sigby&\vdash t_0 : \sigb0\Given\Cb0.
\label{eq:subabs1}
\end{align}
We have assumed a subtype derivation ending in
\[
A;\Exi\vdash
\sigs\Sub\sigx,
(\All{\alpha_1\cdots\alpha_n}\sigay\R\siga0 \Given \Ca0)
\Sub \tau,
C
\]
where
$C\supseteq \Ca0\cup \Cs$.
The preceding is $\alpha$-equivalent to
\[
A;\Exi\vdash
\sigs\Sub\sigx,
(\All{\beta_1\cdots\beta_n}\sigby\R\sigb0 \Given \Cb0)
\Sub \tau,
C.
\]
After $n$ applications of \textsc{S-LE} and one application of
\textsc{S-CE} we obtain
\[
\beta_1,\ldots,\beta_n,A;\Exi\vdash
\sigs\Sub\sigx,
\sigby\R\sigb0 \Sub \tau,
\Cb0\cup C.
\]
\textsc{S-Refl} produces
\begin{align}
\beta_1,\ldots,\beta_n,A;\Exi&\vdash
\sigs\Sub\sigx,
\sigb0\Sub\sigb0,
\sigby\R\sigb0 \Sub \tau,
\Cb0\cup C.
\label{eq:subabs2}
\end{align}
Invoking the induction hypothesis on \eqref{eq:subabs1}~and
\eqref{eq:subabs2} give us derivations for
\def\sig0s{\sigma_0^\s}
\def\C0s{C_0^\s}
\begin{align}
\Gamma,y:\sigby &\vdash t_0[x\mapsto\s] :
\sig0s\Given\C0s,
\label{eq:subabs3}
\\
A';\Exi &\vdash
\sig0s \Sub \sigb0, C'
\label{eq:subabs4}
\end{align}
for some $C'\supseteq\{\sigby\R\sigb0 \Sub \tau\}\cup \Cb0 \cup
C \cup \C0s$. From~\eqref{eq:subabs3}, the desired constrained
typing derivation for $t[x\mapsto\s]$ is derived thus:
%
\infrule[CT-Abs]%
{\emptyset\subseteq
\FTV(\sigby,\sig0s,\C0s)-\FTV(\Gamma)
\andalso
\Gamma,y:\sigby \vdash t_0[x\mapsto\s] :
\sig0s\Given\C0s
}%
{\Gamma\vdash
\Abs y t_0[x\mapsto\s] :
(\sigby\R\sig0s\Given \C0s)\Given\C0s}%
%
From~\eqref{eq:subabs4}, \textsc{S-Refl} and \textsc{S-Dupe} produce
\[
A';\Exi \vdash
\sigby \Sub \sigby,
\sig0s \Sub \sigb0,
\sigby\R\sigb0 \Sub \tau,
C'.
\]
\textsc{S-Arrow} yields
\[
A';\Exi \vdash
\sigby\R\sig0s \Sub \sigby\R\sigb0,
\sigby\R\sigb0 \Sub \tau,
C',
\]
and \textsc{S-Trans} gives us
\[
A';\Exi \vdash
\sigby\R\sig0s \Sub \tau,
C'.
\]
\textsc{S-CI} after duplicating all constraints in $\C0s\subseteq
C'$ produces
\[
A';\Exi \vdash
(\sigby\R\sig0s\Given\C0s) \Sub \tau,
C',
\]
which is the desired subtype derivation.
\end{proof}

\begin{thebibliography}{99}
%\bibitem{Hamkins10}
%Joel David Hamkins
%(\hbox{\url{http://mathoverflow.net/users/1946/joel-david-hamkins}}).\\
%Collection of subsets closed under union and intersection.\\
%URL (version: 2010-01-11):
%\hbox{\url{http://mathoverflow.net/q/11451}}

\bibitem{Boehm85}
Corrado B\"ohm and Alessandro Berarducci.
Automatic synthesis of typed $\Lambda$-programs on term algebras.
\emph{Theoretic Computer Science} 39 (1985).

\bibitem{Laemmel03}
Ralf L\"ammel and Simon Peyton Jones.
Scrap your boilerplate: a practical approach to generic
programming.
\emph{Proceedings of ACM SIGPLAN Workshop on Types in Language
Design and Implementation} (TLDI 2003).

\bibitem{Launchbury94}
John Launchbury and SImon Peyton Jones.
Lazy functional state threads.
\emph{Proceedings of the ACM SIGPLAN 1994 Conference on
Programming Language Design and Implementation} (PLDI 1994).

\bibitem{Milner78}
Robin Milner.
A theory of type polymorphism in programming.
\emph{Journal of Computer and System Sciences} 17 (1978).

\bibitem{Washburn03}
Geoffrey Washburn and Stephanie Weirich.
Boxes go bananas: encoding higher-order abstract syntax with
parametric polymorphism.
\emph{Proceedings of the Eighth ACM SIGPLAN International
Conference on Functional Programming} (ICFP 2003).

\bibitem{Wright94}
Andrew K. Wright and Matthias Felleisen.
A syntactic approach to type soundness.
\emph{Information and Computation} 115-1 (1994).
\end{thebibliography}
\end{document}
