\documentclass{amsart}
\usepackage{bcprules,url,enumerate}
\allowdisplaybreaks
\swapnumbers
\newtheorem{theorem}[subsection]{Theorem}
\newtheorem{lemma}[subsection]{Lemma}
\newtheorem{corollary}[subsection]{Corollary}
\theoremstyle{definition}
\newtheorem{definition}[subsection]{Definition}

\def\thesubsection{\arabic{subsection}}

\title{Nuclear Football Megaslave}

\def\MLF!{ML$^\text{F}$}

\usepackage{stmaryrd}
\usepackage[cmtip,all]{xy}
\newcommand{\nc}{\newcommand}
\newcommand{\DotExpr}[2]{#1 #2.\ }
\newenvironment{syntax}{\[\begin{array}{rrll}}{\end{array}\]}

\makeatletter
\renewcommand{\@secnumfont}{\bfseries}
\makeatother
\def\thesubsection{\arabic{subsection}}

\nc\Abs       {\DotExpr\lambda}
\nc\Add       {\mathrm{add}}
\nc\All       {\DotExpr\forall}
\nc\Allin     {\forall\mathrm{in}}
\nc\Allex     {\forall\mathrm{ex}}
\nc\AbortCC   {\mathrm{abort/cc}}
\nc\ApplyCC   {\mathrm{apply/cc}}
\nc\Bool      {\mathrm{Bool}}
\nc\Brouwer   {\operatorname{Brouwer}}
\nc\Case      {\medbreak\noindent\textbf{Case}~}
\nc\Cps       {\operatorname{cps}}
\nc\Dn[1]     {(#1 \R \bot) \R \bot}
\nc\Dni       {\mathrm{\neg\neg I}}
\nc\Eco       {\mathit{Eco}}
\nc\Env       {\mathit{Env}}
\nc\env       {\mathit{env}}
\nc\Erase     {\mathrm{erase}}
\nc\Ex        {\DotExpr\exists}
\nc\Exin      {\exists\mathrm{in}}
\nc\Exex      {\exists\mathrm{ex}}
\nc\Exmid     {\text{excluded-middle}}
\nc\False     {\mathrm{false}}
\nc\Fix       {\mathrm{fix}}
\nc\FTV       {\operatorname{FTV}}
\nc\FV        {\operatorname{FV}}
\nc\Given     {\text{ given }}
\nc\Halt      {\mathrm{halt}}
\nc\Id        {\mathit{id}}
\nc\If        {\mathrm{if}}
\nc\Ideals    {\mathcal{I}}
\nc\Indent    {\hspace{3em}}
\nc\Int       {\mathrm{Int}}
\nc\Mgsr      {\operatorname{mgsr}}
\nc\LHS       {\operatorname{LHS}}
\nc\Or        {\ | \ }
\nc\Pack      {\mathrm{pack}}
\nc\Piggy     {\mathrm{Piggy}}
\nc\Prune     {\operatorname{prune}}
\nc\Recall    {\DotExpr\Xi} % looks like ∃
\nc\Redo      {\operatorname{redo}}
\nc\RHS       {\operatorname{RHS}}
\nc\Sem[1]    {\llbracket~#1~\rrbracket}
\nc\Sub       {\sqsubseteq}
\nc\Tabs      {\DotExpr\Lambda}
\nc\Tech      {\text{technical}}
\nc\True      {\mathrm{true}}
\nc\R         {\rightarrow}
%\nc\Red       {\xymatrix{{}\ar@{~>}[r]&{}}}
\nc\Red       {\rightsquigarrow}
\nc\RR        {\Rightarrow}
\nc\Unpack    {\mathrm{unpack}}
\nc\Undo      {\operatorname{undo}}
\nc\Unify     {\operatorname{unify}}
\nc\Update[2] {#1\text{ updated }#2}
\nc\Wrong     {\textsc{wrong}}

\begin{document}

\maketitle
\tableofcontents

\subsection{Syntax}
\label{syntax}
\begin{syntax}
\mbox{type}\\
\sigma, \tau
&::=& \iota &\mbox{base type}\\
&|& \alpha & \mbox{type variable} \\
&|& \tau \rightarrow \tau & \mbox{function type} \\
&|& \tau\R\tau\Given C&\mbox{constrained function type}\\
&|& \All\alpha\tau&\mbox{universal type, as long as $\tau$ is}\\
&&&\quad\mbox{neither base type nor type variable}
\\
\\
\mbox{constraints}\\
C
&::=&\emptyset\\
&|& \tau\Sub\tau,C
\\
\\
\mbox{term}\\
s, t & ::= & x & \mbox{variable} \\
& | & \Abs{x:\tau}t  & \mbox{lambda abstraction} \\
& | & t_1~t_2 & \mbox{function application} \\
& | & c~t_1~\cdots~t_n& \mbox{$n$-nary constant}
\end{syntax}

\subsection{Subtyping}

\begin{syntax}
\mbox{list of type variables}\\
A,E
&::=& \emptyset \\
&|& \alpha,A
\\
\\
\mbox{list of constraints}\\
C
&::=& \emptyset \\
&|& \sigma \Sub \tau, C
\\
\\
\mbox{subtype judgement}
&::=&A;E\vdash C
\end{syntax}

\infrule[S-vacuous]
{}
{A;E\vdash\emptyset}

\infrule[S-Refl]
{\FTV(\tau)\subseteq A\cup E
\andalso
A;E\vdash C
}
{A;E \vdash \tau\Sub\tau,C}

% Reason of existence: sublem
\infrule[S-Dupe]
{A;E\vdash \sigma\Sub\tau, C}
{A;E\vdash \sigma\Sub\tau, \sigma\Sub\tau, C}

% Reason of existence: sublem.
\infrule[S-Trans]
{A;E\vdash \tau_0\Sub\tau_1,\tau_1\Sub\tau_2,C}
{A;E\vdash \tau_0\Sub\tau_2,C}

\infrule[S-Arrow]
{A;E\vdash\tau_0\Sub\sigma_0,\sigma_1\Sub\tau_1,C}
{A;E\vdash\sigma_0\R\sigma_1\Sub\tau_0\R\tau_1,C}

\infrule[S-Loner]
{\alpha\notin A\cup E\cup\FTV(C)
\andalso
A;E\vdash
\{ \sigma_i\Sub\tau_j \Or 1 \le i \le m, 1 \le j \le n \}
\cup C
}{
\alpha,A;E \vdash
\sigma_1\Sub\alpha,\ldots,\sigma_m\Sub\alpha,
\alpha\Sub\tau_1,\ldots,\alpha\Sub\tau_n,C
}

\infrule[S-LI]
{\alpha\notin A\cup E\cup\FTV(\tau)\cup\FTV(C)
\andalso
\alpha,A;E\vdash\sigma\Sub\tau,C}
{A;E\vdash(\All\alpha\sigma)\Sub\tau,C}

\infrule[S-LE]
{\alpha\notin A\cup E\cup\FTV(\tau)\cup\FTV(C)
\andalso
A;E\vdash(\All\alpha\sigma)\Sub\tau,C}
{\alpha,A;E\vdash\sigma\Sub\tau,C}

\infrule[S-RI]
{\alpha\notin A\cup E\cup\FTV(\tau)\cup\FTV(C)
\andalso
A;\alpha,E\vdash\sigma\Sub\tau,C}
{A;E\vdash\sigma\Sub(\All\alpha\tau),C}

\infrule[S-CI]
{A;E\vdash \sigma\Sub\tau,C_\sigma\cup C}
{A;E\vdash (\sigma\Given C_\sigma)\Sub\tau,C}

\infrule[S-CE]
{A;E\vdash (\sigma\Given C_\sigma)\Sub\tau,C}
{A;E\vdash \sigma\Sub\tau,C_\sigma\cup C}

Note that there are no analogues of \textsc{S-CI} or
\textsc{S-CE} about constrained types on right hand side of a
subtype constraint. There is the introduction rule \textsc{S-RI}
without a corresponding elimination rule.

\subsection{Constrained typing}
\begin{syntax}
\mbox{constrained typing judgement}
&::=&\Gamma \vdash t : \tau \Given C
\end{syntax}%
\infrule[CT-Var]
{x:\tau\in\Gamma}
{\Gamma\vdash x : \tau \Given \emptyset}

\infrule[CT-Abs]
{
\Gamma,x:\sigma \vdash t : \tau \Given C}
{\Gamma\vdash (\Abs{x:\sigma}t) : \sigma\R\tau \Given C}

\infrule[CT-App]
{
\Gamma\vdash s : \sigma \Given C_1
\andalso
\Gamma\vdash t : \tau \Given C_2}
{\Gamma\vdash s~t : \beta \Given
\sigma\Sub\alpha\R\beta,\tau\Sub\alpha,C_1\cup C_2}

\infrule[CT-Con]
{
\Gamma\vdash s_1:\sigma_1\Given C_1
\andalso
\cdots
\andalso
\Gamma\vdash s_n:\sigma_n\Given C_n
}
{\Gamma\vdash c~s_1~\cdots~s_n:\tau_c\Given
\sigma_1\Sub\tau_1,\ldots,\sigma_n\Sub\tau_n,
{\textstyle\bigcup_{i=1}^n C_i}
}

\infrule[CT-Poly]
{
\alpha_1,\ldots,\alpha_n\notin\FTV(\Gamma)
\andalso
\Gamma\vdash t : \sigma\R\tau \Given C}
{\Gamma\vdash t : (\All{\alpha_1\cdots\alpha_n}\sigma\R\tau\Given C)\Given C}

In \textsc{CT-Con}, the types $\tau_c,\tau_1,\ldots,\tau_n$
depends on $c$ alone. One constant $c$ may have multiple
\textsc{CT-Con} rules, as long as the corresponding
$\delta$-typability requirement is satisfied
(\S\ref{eval}). Here are some examples.

\infrule[CT-Add]
{\Gamma\vdash s_1:\sigma_1\Given C_1
\andalso
\Gamma\vdash s_2:\sigma_2\Given C_2
}
{\Gamma\vdash\Add~s_1~s_2:
\Int\Given\sigma_1\Sub\Int,\sigma_2\Sub\Int,C_1\cup C_2}

\infrule[CT-If]
{\Gamma\vdash s:\sigma\Given C}
{\Gamma\vdash\If~s:\alpha\R\alpha\R\alpha\Given\sigma\Sub\Bool,C}

\infrule[CT-Fix]
{\Gamma\vdash s:\sigma\Given C}
{\Gamma\vdash\Fix~s:\alpha\Given\sigma\Sub \alpha\R\alpha,C}

\subsection{Evaluation}
\label{eval}

\begin{syntax}
\mbox{value}\\
v&::=&c&\mbox{nullary constant of base type}\\
&|&\Abs{x:\sigma}t&\mbox{lambda abstraction}
\end{syntax}

\infrule[$\beta$]
{}
{(\Abs{x:\sigma}s)~t \Red s[x\mapsto t]}

\infrule[$\delta$]
{}
{c~v_1~\cdots~v_n \Red c'}

\infrule[E-App]
{s\Red s'}
{s~t\Red s'~t}

\infrule[E-Con]
{t_i\Red t_i'}
{c~t_1~\cdots~t_i~\cdots~t_n
\Red
c~t_1~\cdots~t_i'~\cdots~t_n
}

The congruence rule \textsc{E-App} implements call-by-name so
that we write the hardest proof. With call-by-value, progress is
easier and preservation is no harder.

The congruence rule \text{E-Con} says that primitive operators
are supposed to be strict, but they are free to force their
arguments in any order.

Rule \textsc{$\delta$} represents all $\delta$-reductions. The
$\delta$-reductions are required to satisfy what
Wright~and Felleisen~\cite{Wright94} call $\delta$-typability:
\begin{align*}
&\text{Let $c$ be the constant in \textsc{CT-Con}.}\\[-4pt]
\tag{$\delta$-typeability}\label{delta-typability}
&\text{For each value $v_i$ with type $\tau_i$,}\\[-4pt]
&\text{there exists a value $v'$ of type $\tau_c$ such that}\\
&\hspace{2em}c~v_1~\cdots~v_n\Red v'.
\end{align*}

By ``nullary constant of base type'', we mean any nullary
constant $c$ for which there is a constrained typing rule
\[
\Gamma\vdash c:\iota\Given\emptyset.
\]
Nullary constants of function type, if exist, are not values. By
$\delta$-typability, each of them evaluates to a lambda
abstraction of appropriate type.

Here are some $\delta$-rules examples.
\begin{align*}
\Add~1~1&\Red2\\
\Add~1~2&\Red3\\
\Add~38701589~8472584&\Red47174173
\end{align*}
One can imagine $\delta$-rules such as
\begin{align*}
\If~\True &\Red\Abs{x:\alpha}\Abs{y:\alpha}x,\\
\If~\False&\Red\Abs{x:\alpha}\Abs{y:\alpha}y,\\
\Fix~(\Abs{x:\sigma} t)&\Red t[x\mapsto \Fix~(\Abs{x:\sigma} t)].
\end{align*}
Proving them to satisfy $\delta$-typability in the precise sense
of preservation (\S\ref{preservation}) is nontrivial, though.

\subsection{Substitution lemma}
\label{sublem}
\def\Xdef{w}
\def\Exigent{\emptyset}
We set out to prove that if $t$ can be typed $\tau_0$, then after
a well-typed substitution it can still be typed $\tau_0$.

Let $\Piggy$ be a set of ``piggy-backed'' constraints. Suppose
the following judgements are derivable.
\begin{align}
\Gamma,x:\sigma_x&\vdash t:\tau\Given C_t \notag
\\
\Gamma&\vdash \Xdef:\sigma_\Xdef \Given C_\Xdef \notag
\\
A;\Exigent&\vdash
\sigma_\Xdef\Sub\sigma_x, \tau\Sub\tau_0,C_t\cup C_\Xdef\cup\Piggy
\label{eq:subass}
\end{align}
Then there exist $A'$, $\tau'$, $C'$ and a set of constraints
$\Piggy'\supseteq\Piggy$ such that we can derive
\begin{align*}
A';\Gamma&\vdash t[x\mapsto \Xdef] : \tau'\Given C',\\
A';\Exigent &\vdash \tau'\Sub\tau_0, C_t\cup C_\Xdef\cup C'\cup\Piggy'.
\end{align*}

\begin{proof}
By induction on a constrained typing derivation of $t$ and by
case analysis on the last rule used.

\Case\textsc{CT-Var}: We know $t=x$. The assumptions are
specialized to the simple form below.
\begin{samepage}
\begin{align}
\Gamma,x:\sigma&\vdash x:\sigma_x\Given\Exigent
\notag
\\
\Gamma&\vdash \Xdef:\sigma_\Xdef\Given C_\Xdef
\label{eq:subvar1}
\\
A;\Exigent&\vdash \sigma_\Xdef\Sub\sigma_x,\sigma_x\Sub\tau_0,C_\Xdef\cup\Piggy
\label{eq:subvar2}
\end{align}
\end{samepage}%
Since $t[x\mapsto \Xdef]=\Xdef$, the desired constrained typing judgement
is exactly \eqref{eq:subvar1}. The desired subtype judgement
follows from one use of \textsc{S-Trans} on \eqref{eq:subvar2}.
\[
A;\Exigent\vdash \sigma_\Xdef\Sub\tau_0,C_\Xdef\cup\Piggy.
\]

\Case\textsc{CT-Abs}: $t=\Abs{y:\sigma_y}s$.
\def\yasump{\Gamma,y:\sigma_y\vdash s:\tau_s\Given C_s}
\infrule[CT-Abs]
{\yasump}
{\Gamma\vdash(\Abs{y:\sigma_y}s):\sigma_y\R\tau_s \Given C_s}
Assumption~\eqref{eq:subass} translates to
\[
A;\Exigent\vdash
\sigma_\Xdef\Sub\sigma_x,
\sigma_y\R\tau_s\Sub\tau_0,C_s\cup C_\Xdef\cup\Piggy.
\]
By \textsc{S-Refl},
\[
A;\Exigent\vdash
    \tau_s\Sub\tau_s,
\sigma_\Xdef\Sub\sigma_x,
\sigma_y\R\tau_s\Sub\tau_0,C_s\cup C_\Xdef\cup\Piggy.
\]
Since the constrained typing of $t$ contains a derivation for
\[\yasump,\]
we invoke the induction hypothesis with the piggyback
$(\sigma_y\R\tau_s\Sub\tau_0,\Piggy)$ to conclude that for some
appropriate $A'$ and argument type annotations,
\begin{align}
s'&=s[x\mapsto \Xdef]\notag
\\
A';\Gamma,y:\sigma_y&\vdash s' : \tau_s'\Given C'
\label{eq:subabs1}
\\
A';\Exigent&\vdash
    \tau_s'\Sub\tau_s,
\sigma_y\R\tau_s\Sub\tau_0,C_s\cup C_\Xdef\cup C'\cup\Piggy',
\label{eq:subabs2}
\end{align}
where $\Piggy'\supseteq\Piggy$. By \eqref{eq:subabs1} and
\textsc{CT-Abs},
\[
A';\Gamma\vdash(\Abs{y:\sigma_y}.s)[x\mapsto \Xdef] : \sigma_y\R\tau_s'\Given C'.
\]
By \eqref{eq:subabs2} and \textsc{S-Refl},
\[
A';\Exigent\vdash
    \sigma_y\Sub\sigma_y,
    \tau_s'\Sub\tau_s,
\sigma_y\R\tau_s\Sub\tau_0,C_s\cup C_\Xdef\cup C'\cup\Piggy'.
\]
\textsc{S-Arrow} produces
\[
A';\Exigent\vdash
    \sigma_y\R\tau_s'\Sub\sigma_y\R\tau_s,
\sigma_y\R\tau_s\Sub\tau_0,C_s\cup C_\Xdef\cup C'\cup\Piggy'
\]
and \textsc{S-Trans} concludes
\[
A';\Exigent\vdash
    \sigma_y\R\tau_s'\Sub\tau_0,
C_s\cup C_\Xdef\cup C'\cup\Piggy'.
\]

\Case\textsc{CT-App}: $t=s_1~s_2$.
\infrule[CT-App]
{
\Gamma\vdash s_1:\sigma_1\Given C_1
\andalso
\Gamma\vdash s_2:\sigma_2\Given C_2
}
{\Gamma\vdash s_1~s_2:\beta\Given
\sigma_1\Sub\alpha\R\beta,\sigma_2\Sub\alpha,C_1\cup C_2
}
Set
\[
C_t=\sigma_1\Sub\alpha\R\beta,\sigma_2\Sub\alpha,C_1\cup C_2,
\]
then assumption~\eqref{eq:subass} becomes
\[
A;\Exigent\vdash
  \sigma_\Xdef\Sub\sigma_x,
  \beta\Sub\tau_0,
  C_t\cup C_\Xdef\cup\Piggy.
\]
Set
\[
D=
\sigma_1\Sub\alpha\R\beta,
\sigma_1\Sub\alpha\R\beta,
\sigma_2\Sub\alpha,
\sigma_2\Sub\alpha,C_1\cup C_2\cup C_\Xdef.
\]
Some uses of \textsc{S-Dupe} and \textsc{S-Refl} produce
\[
A;\Exigent\vdash
  \sigma_1\Sub\sigma_1,
  \sigma_\Xdef\Sub\sigma_x,\sigma_\Xdef\Sub\sigma_x,
  \beta\Sub\tau_0,
  D\cup\Piggy.
\]
Invoking the induction hypothesis on $s_1$, we obtain
\begin{align}
s_1'&=s_1[x\mapsto \Xdef]\notag,
\\
\Gamma&\vdash s_1' : \sigma_1' \Given C_1',
\label{eq:subapp1}
\\
A_1';\Exigent&\vdash
  \sigma_\Xdef\Sub\sigma_x,
  \sigma_1'\Sub\sigma_1,
  \beta\Sub\tau_0,
  D\cup\Piggy_1,
\label{eq:subapp2}
\end{align}
where $\Piggy_1\supseteq\Piggy$. From \eqref{eq:subapp2},
\textsc{S-Refl} produces
\begin{align}
A_1';\Exigent\vdash
  \sigma_2\Sub\sigma_2,
  \sigma_\Xdef\Sub\sigma_x,
  \sigma_1'\Sub\sigma_1,
  \beta\Sub\tau_0,
  D\cup C_1'\cup\Piggy_1.
\label{eq:subappf}
\end{align}
Invoking the induction hypothesis on $s_2$ and \eqref{eq:subapp2}
produces
\begin{align}
A_2';\Exigent&\vdash
  \sigma_1'\Sub\sigma_1,
  \sigma_2'\Sub\sigma_2,
  \beta\Sub\tau_0,
  D\cup C_1'\cup C_2'\cup\Piggy_2
\notag\\
&\vdash
  \beta\Sub\tau_0,
  \sigma_1'\Sub\sigma_1,
  \sigma_2'\Sub\sigma_2,
  % dfn of C_t
  \sigma_1\Sub\alpha\R\beta,\sigma_2\Sub\alpha,
  C_t\cup C_\Xdef\cup C_1'\cup C_2'\cup\Piggy_2,
\notag
\end{align}
the last step by expanding $D$ partially to
$\sigma_1\Sub\alpha\R\beta,\sigma_2\Sub\alpha,C_t\cup C_\Xdef$.
Two uses of \textsc{S-Trans} gives us
\begin{align}
A_2';\Exigent\vdash
  \beta\Sub\tau_0,
  \sigma_1'\Sub\alpha\R\beta,\sigma_2'\Sub\alpha,
  C_t\cup C_\Xdef\cup C_1'\cup C_2'\cup\Piggy_2.
\label{eq:subapps}
\end{align}
Use existing type variables $\alpha$, $\beta$ to construct the
desired constrained typing.
\begin{align}
t'&=s_1[x\mapsto \Xdef]~s_2[x\mapsto \Xdef]
\notag\\
\Gamma&\vdash t':
\beta\Given
\sigma_1'\Sub\alpha\R\beta,\sigma_2'\Sub\alpha,
C_1'\cup C_2'
\notag
\end{align}
The desired subtype judgement is derived already: it is
\eqref{eq:subapps}.

\Case\textsc{CT-Con}: The argument here is a straightforward
generalization of \hbox{\textsc{CT-App}} to $n$ subterms and
arbitrary constraints on the types of subterms.

We know $t=c~s_1~\cdots~s_n$ and $\tau=\tau_c$.
%
\infrule[CT-Con]
{
\Gamma\vdash s_1:\sigma_1\Given C_1
\andalso
\cdots
\andalso
\Gamma\vdash s_n:\sigma_n\Given C_n
}
{\Gamma\vdash c~s_1~\cdots~s_n:\tau_c\Given
\sigma_1\Sub\tau_1,\ldots,\sigma_n\Sub\tau_n,
{\textstyle\bigcup_{i=1}^n C_i}
}
%
Write
\[
C_t=\sigma_1\Sub\tau_1,\ldots,\sigma_n\Sub\tau_n,
{\textstyle\bigcup_{i=1}^n C_i}.
\]
Suppose
\begin{align*}
\Gamma&\vdash s_1[x\mapsto \Xdef]:\sigma_1'\Given C_1',\\
&~\vdots\\
\Gamma&\vdash s_n[x\mapsto \Xdef]:\sigma_n'\Given C_n',\\
\Gamma&\vdash t[x\mapsto \Xdef]:\tau_c\Given
\sigma_1'\Sub\tau_1,\ldots,\sigma_n'\Sub\tau_n,
{\textstyle\bigcup_{i=1}^n C_i'}.
\end{align*}
If we write as usual $t'=t[x\mapsto \Xdef]$, then the constraints
generated in the typing of $t'$ are
\[
C'=\sigma_1'\Sub\tau_1,\ldots,\sigma_n'\Sub\tau_n,
{\textstyle\bigcup_{i=1}^n C_i'}.
\]
By assumption, we can derive
\[
A;\Exigent\vdash \sigma_x\Sub\sigma_\Xdef,\tau_c\Sub\tau_0,
\sigma_1\Sub\tau_1,\ldots,\sigma_n\Sub\tau_n,
({\textstyle\bigcup_{i=1}^n C_i})
\cup C_\Xdef\cup\Piggy.
\]
As in previous cases, some judicious use of \textsc{S-Refl} and
\textsc{S-Dupe} allows us to derive
\[
A;\Exigent\vdash \sigma_x\Sub\sigma_\Xdef,\tau_c\Sub\tau_0,
\sigma_1\Sub\sigma_1,
\sigma_1\Sub\tau_1,\ldots,\sigma_n\Sub\tau_n,
C_t\cup C_\Xdef\cup\Piggy.
\]
The induction hypothesis produces
\[
A_1';\Exigent\vdash \sigma_x\Sub\sigma_\Xdef,\tau_c\Sub\tau_0,
\sigma_1'\Sub\sigma_1,
\sigma_1\Sub\tau_1,\ldots,\sigma_n\Sub\tau_n,
C_1\cup C_t\cup C_\Xdef\cup\Piggy_1,
\]
where $\Piggy_1\supseteq\Piggy$, and \textsc{S-Trans} gives us
\[
A_1';\Exigent\vdash \sigma_x\Sub\sigma_\Xdef,\tau_c\Sub\tau_0,
\sigma_1'\Sub\tau_1,
\sigma_2\Sub\tau_2,\ldots,\sigma_n\Sub\tau_n,
C_1'\cup
C_t\cup C_\Xdef\cup
\Piggy_1.
\]
The same trick will produce
\begin{align*}
A_2';\Exigent\vdash\;& \sigma_x\Sub\sigma_\Xdef,\tau_c\Sub\tau_0,
\sigma_1'\Sub\tau_1,
\sigma_2'\Sub\tau_2,\\
&\sigma_3\Sub\tau_3,
\ldots,\sigma_n\Sub\tau_n,
C_1'\cup C_2'\cup
C_t\cup C_\Xdef\cup\Piggy_2
\end{align*}
all the way up to
\begin{align}
A_n';\Exigent&\vdash \tau_c\Sub\tau_0,
\sigma_1'\Sub\tau_1,\ldots,\sigma_n'\Sub\tau_n,
({\textstyle\bigcup_{i=1}^nC_i'})\cup
C_t\cup C_\Xdef\cup\Piggy_n,
\label{eq:subcon}
\end{align}
refraining from duplicating $\sigma_x\Sub\sigma_\Xdef$ at the last
step. If we collect constraints of $C'$ in \eqref{eq:subcon},
then it becomes the desired subtype judgement
\[
A_n';\Exigent\vdash \tau_c\Sub\tau_0,
C_t\cup C_\Xdef\cup C'\cup\Piggy_n.
\]

\Case\textsc{CT-Poly}: Let $\tau$ be the function type produced
in the constrained typing derivation of $t$.
\infrule[CT-Poly]
{\alpha_1,\ldots,\alpha_n\notin\FTV(\Gamma)\cup\FTV(\sigma_x)
\andalso
\Gamma,x:\sigma_x\vdash t:\tau_\alpha\Given C_\alpha
}
{\Gamma,x:\sigma_x\vdash t:
(\All{\alpha_1\cdots\alpha_n}\tau_\alpha\Given C_\alpha)
\Given C_\alpha}
%
By assumption, we can derive
\begin{align}
A;\Exigent &\vdash \sigma_w\Sub\sigma_x,
(\All{\alpha_1\cdots\alpha_n}\tau_\alpha\Given
C_\alpha)\Sub\tau_0,
C_\alpha\cup C_w\cup\Piggy,
\label{eq:subpoly1}
\end{align}
Choose $\beta_1,\ldots,\beta_n$ outside every type variable
occurring in $\Gamma$ or \eqref{eq:subpoly1}, bound or free or
otherwise. Since all $\alpha_i$ and all $\beta_j$ are disjoint
from free type variables in $\Gamma$ and $\sigma_x$, we can
perform $\alpha$-renaming on the derivation of
\[
\Gamma,x:\sigma_x\vdash t : \tau_\alpha\Given C_\alpha
\]
to obtain a derivation of
\def\tBeta{\Gamma,x:\sigma_x\vdash t : \tau_\beta\Given C_\beta}
\[
\tBeta.
\]
Judgement~\eqref{eq:subpoly1} is $\alpha$-equivalent to
\[
A;\Exigent \vdash \sigma_w\Sub\sigma_x,
(\All{\beta_1\cdots\beta_n}\tau_\beta\Given
C_\beta)\Sub\tau_0,
C_\alpha\cup C_w\cup\Piggy.
\]
Several uses of \textsc{S-LE} followed by \textsc{S-CE} gives us
\[
\beta_1,\ldots,\beta_n,A;\Exigent \vdash \sigma_w\Sub\sigma_x,
\tau_\beta\Sub\tau_0,
C_\alpha\cup C_\beta\cup C_w\cup\Piggy.
\]
Calling the induction hypothesis on $(\tBeta)$ produces
\begin{align}
t&'=t[x\mapsto\Xdef],\notag\\
\Gamma&\vdash t':\tau'\Given C',
\label{eq:subpoly2}
\\
A';\Exigent&\vdash
\tau'\Sub\tau_0,
C_\alpha\cup C_w\cup C'\cup(C_\beta\cup\Piggy'),
\label{eq:subpoly3}
\end{align}
which is exactly what we want, with a bigger piggyback.
\end{proof}

\subsection{Preservation}
\label{preservation}

Informally, if $t$ can be typed $\tau_0$ and $t\Red t'$,
then $t'$ can be given the type $\tau_0$.

Let $t$ be a closed term and let $\Piggy$ be a set of constraints
(on piggyback). If $t\Red t'$ and there exists $A$ such that
\begin{align*}
&\vdash t:\tau \Given C,\\
A;\emptyset&\vdash \tau\Sub\tau_0,C\cup\Piggy,
\end{align*}
then there exist $\tau'$, $A'$, $C'$ and a set of constraints
$\Piggy'\supseteq\Piggy$ such that
\begin{align*}
&\vdash t':\tau'\Given C',\\
A';\emptyset&\vdash \tau'\Sub\tau_0,C\cup C'\cup\Piggy'.
\end{align*}

\begin{proof}
Induction on the derivation $t\Red t'$ and case analysis on the
last rule.

\Case($\beta$): The substitution lemma~(\S\ref{sublem}) gives us
the desired derivations.

\Case($\delta$): Preservation on $\delta$-reduction is the
precise meaning of $\delta$-typability (\S\ref{eval}).

\Case\textsc{E-App}: $t=t_1~t_2$ and $t_1\Red t_1'$. Invoke the
induction hypothesis on $t_1$, piggyback along the two extra
constraints generated by the topmost application of $t$. Make
the conclusion fit the typing of $t'=t_1'~t_2$ by the trick used
at the end of the \textsc{CT-App} case in the proof of the
substitution lemma (\S\ref{sublem}).

\Case\textsc{E-Con}: $t=c~t_1~\cdots~t_i~\cdots~t_n$ and $t_i\Red
t_i'$. Invoke the induction hypothesis on $t_i$, piggyback along
the extra constraints generated by the topmost $n$-nary constant
expression in $t$. Duplicating constraints of the form
$\tau_j\Sub\iota_j$ will give us the desired subtype derivation.
\end{proof}

\subsection{Isolation of base types}
\label{isotypes}
If $\tau$ is not a type variable and not equal to $\iota$, then
it is impossible to derive any subtype judgement of the form
\begin{align*}
A;E&\vdash \iota\Sub\tau,C&
&\mbox{or}&
A;E&\vdash\tau\Sub\iota,C.
\end{align*}
We prove it by induction on the derivation sequence. Due to
\textsc{S-Trans}, we need a stronger induction hypothesis: there
never is any derivation ending in a sequence of constraints
containing a \emph{forbidden chain}
\[
\sigma\Sub\alpha_1,\alpha_1\Sub\alpha_2,\ldots,\alpha_n\Sub\tau,
\]
where $\sigma\neq\tau$, and one of $\sigma,\tau$ is a base type
and the other is not a type variable.

The proof itself is a mechanical case analysis on the last rule
used.

The conclusion of \textsc{S-Vacuous} contains no constraint and
hence no forbidden chain.

If a forbidden chain appears in the conclusion of \textsc{S-Refl},
\textsc{S-Loner}, \textsc{S-Dupe} or \textsc{S-Trans}, then a
forbidden chain exists in the premise, which the induction
hypothesis asserts to be impossible.

If a forbidden chain appears in the conclusion of
\textsc{S-Arrow} or \textsc{S-CI}, then a chain exists in the
premise between a base type and a function type, which is
impossible.

Suppose a forbidden chain appears in the conclusion of
\textsc{S-LI} or \textsc{S-RI}. Since we require polymorphic
types to have a body that is neither base type nor type variable
(\S\ref{syntax}), a chain exists in the premise.

If a forbidden chain appears in the conclusion of \textsc{S-LE}
or \textsc{S-CE}, then there is a forbidden chain in the premise
with a universal type or a constrained function type at one end,
which is impossible.

\subsection{Isolation of base values}
\label{isovalues}
A consequence of \S\ref{isotypes} is as follows. Let $v$ be a
value such that $\Gamma\vdash v:\sigma_v\Given C$. If
\[
A;E\vdash \sigma_v\Sub\iota,C\cup C'
\]
is derivable, then $v$ is a nullary constant. If
\[
A;E\vdash \sigma_v\Sub\sigma\R\tau,C\cup C'
\]
is derivable, then $v$ is a lambda abstraction.

\subsection{Progress} Let $t$ be a closed well-typed term. Which
is to say, there exists a type $\tau_0$ such that for some
$\tau$, $A$, $C$ and $C'$,
\begin{align*}
&\vdash t:\tau\Given C\\
A;\emptyset&\vdash \tau\Sub\tau_0,C\cup C'.
\end{align*}
One of the following is true.
\begin{itemize}
\item $t$ is a value.
\item There exists $t'$ such that $t\Red t'$.
\end{itemize}

\begin{proof}
Induction on the syntax of $t$.

\Case$t=x$: Impossible, since $t$ is closed.

\Case$t=\Abs{x:\sigma}t$: $t$ is a value.

\Case$t=s_0~s_1$: If $s_0\Red s_0'$, then $t\Red s_0'~s_1$. If
$s_0$ does not reduce, then it is a value. We know that
\begin{align}
A;\emptyset&\vdash \beta\Sub\tau_0,
\sigma_0\Sub\alpha\R\beta,\sigma_1\Sub\alpha,C_0\cup C_1,
\label{eq:prog0}
\end{align}
where $\sigma_0\Given C_0$ is the constrained type of $s_0$, and
$\sigma_1\Given C_1$ is the constrained type of $s_1$. Since
\eqref{eq:prog0} is derivable, $s_0$ is a lambda abstraction
(\S\ref{isovalues}), and $t$ is a $\beta$-redex.

\Case$t=c~s_1~\cdots~s_n$: If $n=0$ and $c$ has a base type, then
$t=c$ is a value. If $n>0$ and one of the $s_i$ is reducible,
then $t$ is reducible via the congruence rule \textsc{E-Con}. If
$n>0$ and all $s_i$ are values, then by $\delta$-typability
(\S\ref{eval}), $t$ is a $\delta$-redex.
\end{proof}

\subsection{Faithfulness to System F}

Suppose $\Gamma_F\vdash_F t_F:\tau_F$ in System $F$. Obtain $t$
from $t_F$ by erasing all type abstractions (with appropriate
capture-avoiding $\alpha$-renaming) and all type applications.
If there is a derivation for
\[
\Gamma_F\vdash t : \tau\Given C,
\]
then there is a derivation for
\[
A;\emptyset\vdash \tau\Sub\tau_F, C.
\]

Proof by induction on $\vdash_F$. The CT judgement is not
structurally recursive.

Type abstraction: Use \textsc{CT-Poly} on CT judgement. Recurse.
Done.

Type application: Substitution lemma for System~F types.

Abstraction: Should be easy right? Except \textsc{CT-Poly} may be
called an arbitrary number of times already.

Application: Deal with \textsc{CT-Poly}, recurse,
\textsc{S-Loner}, \textsc{S-Trans}.

Variable: \textsc{S-Refl}.

\subsection{Delta-typability proofs}
Here are $\delta$-typability proofs for concrete constants
$\Add$, $\If$, and $\Fix$, to show that they are possible.

\begin{thebibliography}{99}
%\bibitem{Hamkins10}
%Joel David Hamkins
%(\hbox{\url{http://mathoverflow.net/users/1946/joel-david-hamkins}}).\\
%Collection of subsets closed under union and intersection.\\
%URL (version: 2010-01-11):
%\hbox{\url{http://mathoverflow.net/q/11451}}

\bibitem{Boehm85}
Corrado B\"ohm and Alessandro Berarducci.
Automatic synthesis of typed $\Lambda$-programs on term algebras.
\emph{Theoretic Computer Science} 39 (1985).

\bibitem{Laemmel03}
Ralf L\"ammel and Simon Peyton Jones.
Scrap your boilerplate: a practical approach to generic
programming.
\emph{Proceedings of ACM SIGPLAN Workshop on Types in Language
Design and Implementation} (TLDI 2003).

\bibitem{Launchbury94}
John Launchbury and SImon Peyton Jones.
Lazy functional state threads.
\emph{Proceedings of the ACM SIGPLAN 1994 Conference on
Programming Language Design and Implementation} (PLDI 1994).

\bibitem{Milner78}
Robin Milner.
A theory of type polymorphism in programming.
\emph{Journal of Computer and System Sciences} 17 (1978).

\bibitem{Washburn03}
Geoffrey Washburn and Stephanie Weirich.
Boxes go bananas: encoding higher-order abstract syntax with
parametric polymorphism.
\emph{Proceedings of the Eighth ACM SIGPLAN International
Conference on Functional Programming} (ICFP 2003).

\bibitem{Wright94}
Andrew K. Wright and Matthias Felleisen.
A syntactic approach to type soundness.
\emph{Information and Computation} 115-1 (1994).
\end{thebibliography}
\end{document}
