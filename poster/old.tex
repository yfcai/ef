\def\posterwidth{111}
\def\posterheight{76}

\def\panelwidth{15}
\def\panelheight{10}

\def\prologue{4}
\def\epilogue{3}

\def\puzzlesize{10}
\def\puzzlemargin{1.5*\panelheight}

\def\productionsize{\huge}

\tikzset{every node/.style={inner sep=0pt,outer sep=0pt}}

\begin{textblock}{111}(0,0)
\begin{tikzpicture}
% TITLE
\path(\posterwidth,\posterheight)node[anchor=north east](title){
\pbox{\linewidth}{
\Huge
Title, Author, Contact information, justified at top right
\\
Maybe research question highlight?\\
Don't forget to include a date.
}};


% PROLOGUE
% change this to \path to make invisible
\draw[step=1]
(0, \posterheight - \panelheight)
grid
(\prologue*\panelwidth, \posterheight);


% EPILOGUE
\draw[step=1]
(\posterwidth - \epilogue*\panelwidth, 0)
grid
(\posterwidth, \panelheight);


% CENTRAL PUZZLE
\path
(0.5*\posterwidth, 0.5*\posterheight)
node(central puzzle){
\begin{tikzpicture}
\draw[step=1](0,0)grid(2*\puzzlesize, 2*\puzzlesize);
\end{tikzpicture}
};


% TOP PUZZLE
\path
(0, \posterheight - \puzzlemargin)
node[anchor=north west](top puzzle){
\begin{tikzpicture}
% left piece
\draw[step=1](0,0)grid(\puzzlesize, 2*\puzzlesize);

% right piece
\draw[step=1](2*\puzzlesize,0)grid(3*\puzzlesize, 2*\puzzlesize);
\end{tikzpicture}
};


% BOTTOM PUZZLE
\path
(\posterwidth, \puzzlemargin)
node[anchor=south east](bottom puzzle){
\begin{tikzpicture}
% central piece
\draw[step=1](0,0)grid(2*\puzzlesize, 2*\puzzlesize);

% missing piece
\draw(2*\puzzlesize,0)rectangle(3*\puzzlesize, 2*\puzzlesize);
\end{tikzpicture}
};


% REMARK
\path
(\posterwidth, \posterheight - \puzzlemargin)
node[anchor=north east](remark){
\productionsize
% pbox: varwidth minipage with maximum = \linewidth
\pbox{\linewidth}{
Some stuff about what the ideal model is.\\
The ideal model is idealistic.\\
Maybe an UML diagram? here or at bottom left?
}
%
};


% CORRESPONDENCE
\path
(0, 0)
node[anchor=south west](correspondence){
\productionsize
\pbox{\linewidth}{
The correspondence table between syntax and semantics\\
.\\
.\\
.\\
.\\
.\\
.\\
.\\
and then some.
}
};
\end{tikzpicture}
\end{textblock}
