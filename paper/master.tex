\documentclass{amsart}
\usepackage[foot]{amsaddr}
\usepackage{bcprules,url,verbatim,multicol,caption,fullpage}

\let\phi=\varphi % default phi looks like empty set
\allowdisplaybreaks
\swapnumbers
\newtheorem{theorem}[subsection]{Theorem}
\newtheorem{lemma}[subsection]{Lemma}
\newtheorem{corollary}[subsection]{Corollary}

\def\MLF!{ML$^\text{F}$}

\usepackage{stmaryrd}
\usepackage[cmtip,all]{xy}
\newcommand{\nc}{\newcommand}
\newcommand{\DotExpr}[2]{#1 #2.\ }
\newenvironment{syntax}{\[\begin{array}{rrll}}{\end{array}\]}

\makeatletter
\renewcommand{\@secnumfont}{\bfseries}
\makeatother
\def\thesubsection{\arabic{subsection}}

\nc\Abs       {\DotExpr\lambda}
\nc\Add       {\mathrm{add}}
\nc\All       {\DotExpr\forall}
\nc\Allin     {\forall\mathrm{in}}
\nc\Allex     {\forall\mathrm{ex}}
\nc\AbortCC   {\mathrm{abort/cc}}
\nc\ApplyCC   {\mathrm{apply/cc}}
\nc\Bool      {\mathrm{Bool}}
\nc\Brouwer   {\operatorname{Brouwer}}
\nc\Case      {\medbreak\noindent\textbf{Case}~}
\nc\Cps       {\operatorname{cps}}
\nc\Dn[1]     {(#1 \R \bot) \R \bot}
\nc\Dni       {\mathrm{\neg\neg I}}
\nc\Eco       {\mathit{Eco}}
\nc\Env       {\mathit{Env}}
\nc\env       {\mathit{env}}
\nc\Erase     {\mathrm{erase}}
\nc\Ex        {\DotExpr\exists}
\nc\Exin      {\exists\mathrm{in}}
\nc\Exex      {\exists\mathrm{ex}}
\nc\Exmid     {\text{excluded-middle}}
\nc\False     {\mathrm{false}}
\nc\Fix       {\mathrm{fix}}
\nc\FTV       {\operatorname{FTV}}
\nc\FV        {\operatorname{FV}}
\nc\Given     {\text{ given }}
\nc\Halt      {\mathrm{halt}}
\nc\Id        {\mathit{id}}
\nc\If        {\mathrm{if}}
\nc\Ideals    {\mathcal{I}}
\nc\Indent    {\hspace{3em}}
\nc\Int       {\mathrm{Int}}
\nc\Mgsr      {\operatorname{mgsr}}
\nc\LHS       {\operatorname{LHS}}
\nc\Or        {\ | \ }
\nc\Pack      {\mathrm{pack}}
\nc\Piggy     {\mathrm{Piggy}}
\nc\Prune     {\operatorname{prune}}
\nc\Recall    {\DotExpr\Xi} % looks like ∃
\nc\Redo      {\operatorname{redo}}
\nc\RHS       {\operatorname{RHS}}
\nc\Sem[1]    {\llbracket~#1~\rrbracket}
\nc\Sub       {\sqsubseteq}
\nc\Tabs      {\DotExpr\Lambda}
\nc\Tech      {\text{technical}}
\nc\True      {\mathrm{true}}
\nc\R         {\rightarrow}
%\nc\Red       {\xymatrix{{}\ar@{~>}[r]&{}}}
\nc\Red       {\rightsquigarrow}
\nc\RR        {\Rightarrow}
\nc\Unpack    {\mathrm{unpack}}
\nc\Undo      {\operatorname{undo}}
\nc\Unify     {\operatorname{unify}}
\nc\Update[2] {#1\text{ updated }#2}
\nc\Wrong     {\textsc{wrong}}

\begin{document}\thispagestyle{empty}

% 'ware that blank page
%\newgeometry{left=1.5cm,right=1.5cm}


\def\CT0{$\text{CT}_0$}
\def\DoCap{\captionsetup{justification=raggedright,margin=0pt}}

\def\OK{\\[5pt]}

\def\LeftWidth{0.5\textwidth}
\def\RightWidth{0.5\textwidth}

{
\begin{minipage}{\LeftWidth}
\begin{tabular}{ll}
Name & Yufei Cai\OK
Affiliation & Philipps-Universit\"at Marburg\OK
Email & \url{cai@mathematik.uni-marburg.de} \OK
Address & 05D06 Mehrzweckgeb\"aude,\\
              & Hans-Meerwein Stra\ss e,\\
              & 35039 Marburg, Germany \OK
Advisor & Prof.\ Dr.\ Klaus Ostermann \OK
ACM number & 9866679 \OK
Category & Graduate student \OK
Research title & An extensional viewpoint\\ &of polymorphism
\end{tabular}

\vspace{2.3cm}
\small
\begin{verbatim}
{-# LANGUAGE RankNTypes,
    ScopedTypeVariables #-}
import Prelude hiding (sum)

type List a =
  forall b. b -> (a -> b -> b) -> b

nil :: List a
nil = const

cons :: a -> List a -> List a
cons x xs = \z s -> s x (xs z s)

type MatchList a =
  forall c. c -> (a -> List a -> c) -> c

matchList :: List a -> MatchList a
matchList xs =
  xs const
     (\x (ys :: MatchList a) z s ->
       s x (ys nil cons))

sum :: List Int -> Int
sum xs = matchList xs 0 (\x xs -> x + sum xs)

downFrom :: Int -> List Int
downFrom 0 = nil
downFrom n = cons n (downFrom (n - 1))

main = putStrLn (show (sum (downFrom 100)))
\end{verbatim}

\begingroup
\DoCap
\captionof{figure}{B\"ohm-Berarducci encoding of lists\\
in ill-typed Haskell. The compiler complains about the definition
of \texttt{matchList}.}
\label{figure1}
\endgroup

\bigbreak




\vfil
\end{minipage}
\begin{minipage}{\RightWidth}
\renewenvironment{syntax}{\[\begin{array}{rrlr}}{\end{array}\]}
\def\Separate{\vspace{0.4cm}}

\small
\begin{syntax}
t&::=& c \Or x \Or t~t \Or \Abs x t
&\mbox{term}\\
&|& \Abs {x : \tau} t
\OK
\tau &::=& \iota \Or \alpha \Or \tau \R \tau 
\Or \All\alpha\tau&\mbox{type}\\
&|&\tau\Given C&\mbox{(internal use only)}
\OK
C&::=&\emptyset \Or \tau\Sub\tau,C&\mbox{constraints}
\end{syntax}

\Separate

\infrule[CT-Con]
{}
{\Gamma\vdash c : \tau_c \Given \emptyset}

\infrule[CT-Var]
{x:\tau\in\Gamma}
{\Gamma\vdash x : \tau \Given \emptyset}

\infrule[CT-App]
{
\Gamma\vdash s : \sigma \Given C_1
\andalso
\Gamma\vdash t : \tau \Given C_2}
{\Gamma\vdash s~t : \beta \Given
\sigma\Sub\alpha\R\beta,\tau\Sub\alpha,C_1\cup C_2}

\infrule[CT-Abs]
{
\begin{array}{c}
\{\alpha_1,\ldots,\alpha_n\}
=
\FTV(\sigma,\tau,C)-\FTV(\Gamma)
\\
\alpha \notin \FTV(\Gamma)
\andalso
\Gamma,x:\alpha \vdash t : \tau \Given C
\end{array}}
{\Gamma\vdash (\Abs {x} t) :
(\All{\alpha_1\cdots\alpha_n}\alpha\R\tau\Given C) \Given C}

\infrule[CT-Hint]
{
\begin{array}{c}
\{\alpha_1,\ldots,\alpha_n\}
=
\FTV(\sigma,\tau,C)-\FTV(\Gamma)
\\
\Gamma,x:\sigma \vdash t : \tau \Given C
\end{array}}
{\Gamma\vdash (\Abs {x:\sigma} t) :
(\All{\alpha_1\cdots\alpha_n}\sigma\R\tau\Given C) \Given C}

\Separate

\begin{syntax}
A,E &::=&\emptyset \Or \alpha,A &\qquad\qquad\mbox{lists of type variables}
\end{syntax}

\infrule[S-vacuous]
{}
{A;E\vdash\emptyset}

\infrule[S-Refl]
{
A;E\vdash C
}
{A;E \vdash \tau\Sub\tau,C}

\infrule[S-Arrow]
{A;E\vdash\tau_0\Sub\sigma_0,\sigma_1\Sub\tau_1,C}
{A;E\vdash\sigma_0\R\sigma_1\Sub\tau_0\R\tau_1,C}

\infrule[S-LI]
{
\begin{array}{cc}
\tau\notin A\\
\alpha\text{ is only free in $\sigma$}
&\andalso
\alpha,A;E\vdash\sigma\Sub\tau,C
\end{array}
}
{A;E\vdash(\All\alpha\sigma)\Sub\tau,C}

\infrule[S-RI]
{
\begin{array}{cc}
\sigma\notin A\\
\alpha\text{ is only free in $\tau$}
&\andalso
A;\alpha,E\vdash\sigma\Sub\tau,C
\end{array}
}
{A;E\vdash\sigma\Sub(\All\alpha\tau),C}

\infrule[S-Loner]
{
\alpha\text{ fresh}
\andalso
A;E\vdash
\{ \sigma_i\Sub\tau_j \Or 1 \le i \le m, 1 \le j \le n \}
\cup C
}{
\alpha,A;E \vdash
\sigma_1\Sub\alpha,\ldots,\sigma_m\Sub\alpha,
\alpha\Sub\tau_1,\ldots,\alpha\Sub\tau_n,C
}


\begingroup
\DoCap
\captionof{figure}{Syntax and typing rules of \CT0}
\label{figure2}
\endgroup
\end{minipage}
}

\vfil
\break

%\restoregeometry

\def\Fig#1{Figure~\ref{figure#1}}
\def\fig#1{figure~\ref{figure#1}}

\normalsize

\subsection{Problem statement}

We want to capture the expressive power of System~F in a
practical programming language.
In System~F,
\begin{enumerate}
\item we can write polymorphic functions, and
\item we can take functions definable via (1) or (2) as arguments.
\end{enumerate}
Call this feature ``first-class polymorphism''. It is a powerful
abstraction mechanism, essential in unrestricted use of design
patterns like scrap-your-boilerplate~\cite{Laemmel03} and capable
of encoding algebraic datatypes~\cite{Boehm85}. \Fig1 shows the
encoding of lists. The synonym~\texttt{List} expands to a type of
polymorphic functions. The synonym \texttt{MatchList} encodes
pattern-matching on lists, and expands to a type of polymorphic
functions whose second argument is a function whose second
argument is a polymorphic function.

\Fig1 is ill-typed in Haskell. However, if we insert enough type
casts (\texttt{unsafeCoerce}) into the definition of
\texttt{matchList}, it will compile, and the resulting executable
prints \texttt{5050} as expected. In fact, this program can never
encounter runtime type errors. A type system is desirable if it
accepts \fig1 without further ado, and rejects all programs
concealing runtime type errors.

\subsection{Background} Researchers recognized the importance of
first-class polymorphism and spent much effort toward making it
practical. Many systems were proposed that offer the expressive
power of System~F with less notational overhead. They can be
\emph{predicative} or \emph{impredicative}.

Like F, a \emph{predicative} system is based on instantiating
type variables with type arguments. But if a type argument is not
polymorphic (i.~e., it is a \emph{monotype}), then it can be
inferred. Practical use cases tend to contain polymorphic type
arguments; the user must supply those in some way. Haskell's
Rank-N-Types is a representative~\cite{Jones07}. (The reader may
try annotating \fig1 until Rank-N-Types accepts it without
\texttt{unsafeCoerce}.)

An \emph{impredicative} system will infer a type argument even if
it contains quantifiers. Since inferring all type abstractions
and type applications is undecidable for F~\cite{Schubert98}, an
impredicative system can choose to not infer everything, or
accept more programs than System~F does. FPH~\cite{Vytiniotis08}
is an example of the former. \MLF!~\cite{LeBotlan03,RemyMLF} and
HML~\cite{Leijen09} are examples of the latter; they extend F
types with bounded quantification, where the quantified variable
is constrained against an arbitrary type. \MLF! rejects \fig1 but
accepts it after removing \emph{all} type annotations and
signatures---it interprets F types somewhat differently. I do not
know whether HML accepts \fig1, because its prototype has
technical problems processing \texttt{matchList}. \MLF! and HML
are harder to implement than our system \CT0, shown in \fig2.

\subsection{Approach}

To my knowledge, all existing systems are \emph{intentional}: A
type is a syntactic object or a collection of syntactic objects.
Instantiation of type variables is an act of substitution.

The effort toward practical first-class polymorphism may benefit
from an \emph{extensional} viewpoint: Functions are understood in
terms of their behavior, as argument-result pairs. Polymorphism
arises when a function passes information unaltered from
contravariant positions to covariant positions. Types encode
statements about the behavior of functions; they contain demands
for polymorphism and promises of it. This view leads to the
design of constrained type system version 0 (\CT0), especially of
the rule \textsc{S-Loner} (\fig2).

\subsection{Results and future work}

\CT0 accepts \fig1 (in lambda calculus syntax) and rejects all
programs with runtime type errors. We tested \CT0 on well-typed F
terms of depth $\le5$ with type abstractions and type
applications erased; all were accepted.

\CT0 is easy to implement. To check whether a term $t$ has the
closed type $\tau$:
\begin{enumerate}
\item Descend into $t$ and use CT-rules to obtain a constrained
type $(\sigma\Given C)$.
\item Let $A$ be the set of type variables free in $\sigma$ or
$C$. Attempt to construct a derivation of
$(A;\emptyset\vdash\sigma\Sub\tau,C)$ by applying S-rules
backwards greedily and exhaustively. If it succeeds, then $t$ has
type $\tau$.
\end{enumerate}

For future work, I hope to formalize the extensional intuition
behind \CT0. It would give us another tool to reason about
polymorphism. It might also answer questions about the practical
viability of \CT0, such as whether all System~F terms are
well-typed in \CT0 after erasure, and when exactly should an
argument be annotated.

\newpage\normalsize
\begin{thebibliography}{99}
%\bibitem{Hamkins10}
%Joel David Hamkins
%(\hbox{\url{http://mathoverflow.net/users/1946/joel-david-hamkins}}).\\
%Collection of subsets closed under union and intersection.\\
%URL (version: 2010-01-11):
%\hbox{\url{http://mathoverflow.net/q/11451}}

\bibitem{Boehm85}
Corrado B\"ohm and Alessandro Berarducci.
Automatic synthesis of typed $\Lambda$-programs on term algebras.
\emph{Theoretic Computer Science} 39 (1985).

\bibitem{Laemmel03}
Ralf L\"ammel and Simon Peyton Jones.
Scrap your boilerplate: a practical approach to generic
programming.
\emph{Proceedings of ACM SIGPLAN Workshop on Types in Language
Design and Implementation} (TLDI 2003).

\bibitem{Launchbury94}
John Launchbury and SImon Peyton Jones.
Lazy functional state threads.
\emph{Proceedings of the ACM SIGPLAN 1994 Conference on
Programming Language Design and Implementation} (PLDI 1994).

\bibitem{Milner78}
Robin Milner.
A theory of type polymorphism in programming.
\emph{Journal of Computer and System Sciences} 17 (1978).

\bibitem{Washburn03}
Geoffrey Washburn and Stephanie Weirich.
Boxes go bananas: encoding higher-order abstract syntax with
parametric polymorphism.
\emph{Proceedings of the Eighth ACM SIGPLAN International
Conference on Functional Programming} (ICFP 2003).

\bibitem{Wright94}
Andrew K. Wright and Matthias Felleisen.
A syntactic approach to type soundness.
\emph{Information and Computation} 115-1 (1994).
\end{thebibliography}
\end{document}
