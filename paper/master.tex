\documentclass{amsart}
\usepackage[foot]{amsaddr}
\usepackage{bcprules,url,verbatim,multicol,enumerate}

\let\phi=\varphi % default phi looks like empty set
\allowdisplaybreaks
\swapnumbers
\newtheorem{theorem}[subsection]{Theorem}
\newtheorem{lemma}[subsection]{Lemma}
\newtheorem{corollary}[subsection]{Corollary}

\def\MLF!{ML$^\text{F}$}

\usepackage{stmaryrd}
\usepackage[cmtip,all]{xy}
\newcommand{\nc}{\newcommand}
\newcommand{\DotExpr}[2]{#1 #2.\ }
\newenvironment{syntax}{\[\begin{array}{rclr}}{\end{array}\]}

\makeatletter
\renewcommand{\@secnumfont}{\bfseries}
\makeatother
\def\thesubsection{\arabic{section}.\arabic{subsection}}
\def\+={\text{+=}}
\setlength\fboxsep{0pt}

\nc\Abs       {\DotExpr\lambda}
\nc\Add       {\mathrm{add}}
\nc\All       {\DotExpr\forall}
\nc\Allin     {\forall\mathrm{in}}
\nc\Allex     {\forall\mathrm{ex}}
\nc\AbortCC   {\mathrm{abort/cc}}
\nc\ApplyCC   {\mathrm{apply/cc}}
\nc\B         {\mathbb}
\nc\Bool      {\mathrm{Bool}}
\nc\Bot       {\texttt{Bot}}
\nc\BoxIt[1]  {\framebox{\raisebox{0.1ex}[1.5ex][0.2ex]{$#1$}}}
\nc\Brouwer   {\operatorname{Brouwer}}
\nc\Case      {\medbreak\noindent\textbf{Case}~}
\nc\Cap       {\MakeL\cap}
\nc\Closure   {\mathrm{closure}}
\nc\Cps       {\operatorname{cps}}
\nc\Cup       {\MakeL\cup}
\nc\Dn[1]     {(#1 \R \bot) \R \bot}
\nc\Dni       {\mathrm{\neg\neg I}}
\nc\Dom       {\mathrm{dom}}
\nc\Down      {\mathord\downarrow}
\nc\Eco       {\mathit{Eco}}
\nc\Env       {\mathit{Env}}
\nc\env       {\mathit{env}}
\nc\Erase     {\mathrm{erase}}
\nc\Ex        {\DotExpr\exists}
\nc\Exin      {\exists\mathrm{in}}
\nc\Exex      {\exists\mathrm{ex}}
\nc\Exmid     {\text{excluded-middle}}
\nc\False     {\mathrm{false}}
\nc\Fix       {\mathrm{fix}}
\nc\FTV       {\operatorname{FTV}}
\nc\FV        {\operatorname{FV}}
\nc\Given     {\text{ given }}
\nc\Halt      {\mathrm{halt}}
\nc\Id        {\mathit{id}}
\nc\If        {\mathrm{if}}
\nc\Ideals    {\mathcal{I}}
\nc\Indent    {\hspace{3em}}
\nc\Int       {\texttt{Int}}
\nc\Inject    {\mathrm{inject}}
\nc\JCT       {J_{\rm CT}}
\nc\JF        {J_{\rm F}}
\nc\JS        {J_{\rm S}}
\nc\LHS       {\operatorname{LHS}}
\nc\Mgsr      {\operatorname{mgsr}}
\nc\MakeT[1]  {\ \BoxIt{#1}\ }
\nc\MakeL[1]  {\ \mathrm{:\hspace{-2pt}#1\hspace{-2pt}:}\ }
\nc\Or        {\ | \ }
\nc\Pack      {\mathrm{pack}}
\nc\Par       {\subsection}
\nc\Piggy     {\mathrm{Piggy}}
\nc\Plus      {\MakeT+}
\nc\Prune     {\operatorname{prune}}
\nc\Rank      {\mathrm{rank}}
\nc\Recall    {\DotExpr\Xi} % looks like ∃
\nc\Redo      {\operatorname{redo}}
\nc\RHS       {\operatorname{RHS}}
\nc\Sem[1]    {\SemSlim{~#1~}}
\nc\SemSlim[1]{\llbracket#1\rrbracket}
\nc\String    {\texttt{String}}
\nc\Sub       {\sqsubseteq}
\nc\Sup       {\sqsupseteq}
\nc\Tabs      {\DotExpr\Lambda}
\nc\Tech      {\text{technical}}
\nc\Times     {\MakeT\times}
\nc\True      {\mathrm{true}}
\nc\Type      {\mathbf{Type}}
\nc\R         {\rightarrow}
\nc\RL        {\MakeL{\BoxIt\R}}
\nc\RT        {\MakeT\R}
%\nc\Red       {\xymatrix{{}\ar@{~>}[r]&{}}}
\nc\Red       {\rightsquigarrow}
\nc\RR        {\Rightarrow}
\nc\Top       {\texttt{Top}}
\nc\Unpack    {\mathrm{unpack}}
\nc\Undo      {\operatorname{undo}}
\nc\Unify     {\operatorname{unify}}
\nc\Up        {\mathord{\uparrow}}
\nc\Update[1] {#1\Updated}
\nc\Updated   {\text{ updated }}
\nc\VST       {\vdash_{\textrm{ST}}}
\nc\VSub      {\vdash_{\textrm S}}
\nc\Wrong     {\textsc{wrong}}


\def\CT0{$\text{CT}_0$}

\title[The MPS model of possibly negative algebraic data types]
{The MacQueen-Plotkin-Sethi model of
\\
possibly negative algebraic data types}

\begin{document}

\maketitle

\section{Roadmap}

\Par*{Problem}

\begin{itemize}
\item Why do you want to apply lambda models to type checking?
\item
Why do you want type systems that are sound by construction?
\item
Why do you want lots of variations in type systems?
\item
What is the advantage of a semantic-first approach to designing
type systems?
\end{itemize}

Given a model, there is a guide to designing typing rules. Each
one can be made a tautology of that model. Any type system formed
by a collection of such tautologies is sound by construction. One
can add typing rules without fear of destroying soundness.

\Par*{Solution}
\begin{enumerate}
\item Fix a semantic domain for untyped lambda calculus.
\item Define types as certain sets of the domain.
\item Come up with a type system that checks whether the value
denoted by a lambda term belong to some type. As long as every
typing rule describes a tautology about types as sets in the
semantic domain, the resulting system is sound by construction.
\end{enumerate}

\Par*{More problems}

\begin{itemize}
\item How do you reason about decidability of type checking?
\end{itemize}

This paper demonstrates one sound-by-construction type system
based on the MPS model. It must answer these questions.
\begin{enumerate}
\item What is the MPS model?
\item What are the tautological typing rules?
\item How does it perform in practice?
\end{enumerate}
We can write (1) and (2) for now. Deal with (3) when the time comes.

\section{What is a type?}

MacQueen, Plotkin and Sethi proposed a notion of types as sets in
a semantic domain of untyped lambda calculus. These types are
rough descriptions of the behavior of their inhabitants. For
example, if a function $f$ has type $\mathbb Z\R\mathbb Z$, then
$f$ is guaranteed to map an integer to another integer. If $g$
has type $\All\alpha \alpha\R\alpha$, then $g(x)$ inhabits every
type where $x$ is an inhabitant. Under this scheme, a recursive
type equation such as
\[
\mathit{List} = \{\mathit{nil}\} + (\mathbb Z \times \mathit{List})
\]
has a unique solution, and thereby defines a type.

In the MacQueen-Plotkin-Sethi model, a type is a nonempty
Scott-closed subset of the value domain $V$. The existence and
uniqueness proof for solutions of recursive type equations
depends crucially on the properties of Scott-closed sets.


\Par{Value domain}

$V$ is a set containing all functions definable in untyped lambda
calculus. It is the solution of the following equation up to
set-theoretic isomorphism:
\begin{equation}\label{domain-eq}
V \cong B + (V + V) + (V \times V) + (V \R V) + \{\bot, \top\}.
\end{equation}
Here, $B$ is any set of base values. In a practical programming
language, its members are often truth values, integers and
floating point numbers. The symbol $+$ denotes disjoint union,
and $\times$ denotes cartesian product. They give rise to a
\emph{approximation} partial order $\Sub$ under which $\bot$ is
always the least element, $\top$ is always the greatest element,
and members of $B$ are incomparable with one another. The
approximation partial order $\Sub$ determines a Scott topology on
$V$, and the function space $(V \R V)$ consists of the continuous
functions of the Scott topology. We will talk about Scott
topologies later. For now, it suffices to appreciate that $V$
contains base values, function values and values of (not
necessarily positive) algebraic data types.

There are a number of ways to construct a set $V$ satisfying
equation~\eqref{domain-eq}. We use the construction producing a
\emph{consistently complete algebraic cpo}. We will use the
property of $V$ as a cpo in the definition of types, and use the
algebraicity of $V$ to establish that recursive algebraic data
types are well-defined. Consistent completeness is unused in our
development.

Previous works define an extra member $\Wrong$ of $V$ to stand
for runtime type errors. We use the maximum element $\top$ for
runtime type errors instead, so that $\Wrong$ does not have to be
purposefully excluded from every type. In fact, our types are
closed nonempty proper subsets of $V$ under its Scott topology.

\Par{Complete partial orders}

Let $\sqsubseteq$ be a partial order over $V$. A subset of $V$ is
\emph{directed} if every two members $x$, $y$ of $A$ has an upper
bound $z$ in $A$ such that $z\sqsupseteq x$ and $z\sqsupseteq y$.
The partially ordered set $V$ is a \emph{complete partial order},
or \emph{cpo}, if every directed subset of $V$ has a supremum
(i.~e., least upper bound) in $V$.

Our chosen solution of equation~\eqref{domain-eq} is a cpo. We
will henceforth use that fact without further qualification.

\Par{Scott topology}

The solution to domain equation~\eqref{domain-eq} should impose a
\emph{approximation} partial order $\Sub$ on $V$ satisfying the
following properties.
\begin{itemize}
\item $\bot$ is the least element: $\bot\Sub v$ for all $v\in V$.
\item $\top$ is the greatest element: $v\Sub\top$ for all $v\in
V$.
\item Base values are incomparable: If $u,v\in B$, then
$u\not\Sub v$ and $v\not\Sub u$.
\end{itemize}
A topology of $V$ is any labeling of subsets of $V$ as
\emph{open} such that
\begin{itemize}
\item $\emptyset$ and $V$ are open,
\item the union of any family of open sets is open,
\item the intersection of a finite number of open sets is open.
\end{itemize}
We mentioned that the function space $(V\R V)$ consists of the
continuous functions according to the Scott topology. A function
is continuous if its preimage of each open set is an open set.

\begin{samepage}
The Scott topology of $V$ labels a subset $S$ open if it
satisfies the following conditions.
\begin{itemize}
\item $S$ is \emph{upward-closed}: If $u\in S$ and $u\Sub v$,
then $v\in S$.
\item $S$ is \emph{inaccessible by directed supremums}: If $A$ is
a directed set disjoint from $S$, then the supremum of $A$ is
outside $S$.
\end{itemize}
\end{samepage}

A set is \emph{closed} if it is the complement of an open set.

\begin{lemma}[Characterizing closed sets of Scott topology]
A set $S\subseteq V$ is closed under Scott topology if and
only if it satisfies the following conditions.
\begin{itemize}
\item $S$ is \emph{downward-closed}: If $v\Sub u$ and $u\in S$,
then $v\in S$.
\item $S$ is \emph{closed under directed supremums}: If
$A$ is a directed subset of $S$, then the supremum of $A$ is a
member of $S$.
\end{itemize}
\end{lemma}

\begin{proof}
$S$ is downward-closed if and only if its complement is
upward-closed. $S$ is closed under directed supremums if and only
if its complement is inaccessible by directed supremums.
\end{proof}

\Par{Types}

A subset of $V$ is a \emph{type} if it is a proper closed set
under the Scott topology. In other words, types are nonempty
proper subsets of $V$ closed downward and under directed
supremums. If $v\in T$, then we say that the value $v$
\emph{inhabits} the type $T$, and that $v$ is an
\emph{inhabitant} of $T$.

We will use the property of types as closed sets to establish
well-approximation of recursive algebraic data types.

\begin{lemma}
$\bot$ inhabits every type. $\top$ does not inhabit any type.
\end{lemma}

\section{Recursively defined types}

\Par{Road map}

Let us reproduce the result by MacQueen, Plotkin and Sethi that
types can be defined by recursive equations. For example, the
type of heterogeneous lists is defined by the equation
\[
\texttt{List} = \texttt{Unit} + (V \times \texttt{List}),
\]
which holds with set-theoretic identity. The argument has 3
steps.
\begin{enumerate}[(i)]
\item Define the distance between every pair of types, and
establish completeness of the resulting metric space in the sense
that every Cauchy sequence of types converges to a type.
\item Show that the product, sum and function type constructions
are contractive in both arguments,
\item Invoke the Banach fixed-point theorem on the nonempty
complete metric space of types to show that every contraction
mapping between types has a unique fixed point. By (2), a
recursive type equation defines a unique type if its right hand
side uses only the product, sum and function type constructions.
\end{enumerate}

Step~(i) works for every distance function in a certain family.
But step~(ii) works for only one particular distance function,
whose definition depends crucially on the properties of $V$ as an
algebraic cpo.

\Par{Compact elements and algebraic cpos}

An element $v\in V$ is \emph{compact} if for all directed set $A$
such that $v\Sub\sup A$, there exists $u\in A$ such that $v\Sub
u$.

For any subset $S\subseteq V$, write $S_c$ for the set of compact
elements in $S$. A cpo $V$ is \emph{algebraic} if every $v\in V$
is the supremum of smaller compact elements:
\[
v=\sup \{u \in V_c \Or u \Sub v \}.
\]
Our chosen solution of equation~\eqref{domain-eq} is an algebraic
cpo. We will use its algebraicity without ceremony from now on.

\begin{lemma}
Let $V$ be an algebraic cpo. A subset $S$ contains every compact
element, or $S\supseteq V_c$, if and only if for every $v\in V$,
\[
v=\sup\{u\in S \Or u\Sub v\}.
\]
\end{lemma}

\begin{proof}
($\Rightarrow$) By algebraicity and because $V_c\subseteq S$,
\[
v=\sup\{u\in V_c \Or u \Sub v\} \Sub \sup\{u\in S \Or u \Sub v\}.
\]
Since $v$ is an upper bound of $\{u\in S \Or u \Sub v\}$, we also
have
\[
v\Sup \sup\{u\in S \Or u \Sub v\}.
\]
The proof goal follows from antisymmetry of the approximation
partial order $\Sub$.

($\Leftarrow$) Choose arbitrary $v\in V_c$. We are to show $v\in
S$. Let $A= \{u \in S \Or u \Sub v\}$. By assumption $v=\sup A$.
Since $v$ is compact, there exists $u'\in A$ such that $v\Sub
u'$. But we also have $u'\Sub v$ by definition of $A$, which
implies $v=u'\in A\subseteq S$.
\end{proof}

\Par{Rank}

\begin{lemma}
Every compact element has a rank.
\end{lemma}

\Par{Converging sequences of sets}
Let us recall the standard definition of convergence.

The \emph{limit superior} of an infinite sequence of sets
$S_1,S_2,\ldots$ is
\[
\limsup_{n\rightarrow\infty}S_n =
\bigcap_{n=1}^\infty\bigcup_{i = n}^\infty S_i.
\]
The \emph{limit inferior} is
\[
\liminf_{n\rightarrow\infty}S_n =
\bigcup_{n=1}^\infty\bigcap_{i = n}^\infty S_i.
\]
If the limit superior and limit inferior are equal, then the
sequence $S_1,S_2,\ldots$ \emph{converges}, and its \emph{limit}
is
\[
S = \limsup_{n\rightarrow\infty}S_n = \liminf_{n\rightarrow\infty}S_n.
\]

\Par{Metric space of types}

Let ``$\Rank$'' be an arbitrary function from $V$ to $\mathbb R$.
The proximity of two types $S$, $T$ is the smallest rank of a
value in the symmetric difference of $S$ and $T$. If $S=T$, then
their proximity is $\infty$. The distance $d(S, T)$ between two
types is inverse exponential in their proximity:
\[
d(S,T) =
\left(\frac12\right)^{\mathrm{proximity}(S, T)}
%\frac1{2 ^{\mathrm{proximity}(S, T)}} % looks worse
 \]
It is easy to verify that the distance function $d$ satisfies the
requirements for forming a metric space over the set of types:
It is nonnegative, evaluates to $0$ only on equal types, is
commutative, and satisfies a stronger inequality than the
triangle inequality:
\[
d(R,T)\le\max(d(R,S),d(S,T)).
\]
To see it, let $v$ be the value of minimum rank $r$ in the
symmetric difference between $R$~and $T$, and suppose $v\in R-T$.
Then $d(R, T)=2^{-r}$. If $v\in S$, then $v\in S-T$ and
$d(S,T)\ge2^{-r}$. If $v\notin S$, then $d(R, S)\ge2^{-r}$.

\Par{Cauchy sequences}

An infinite sequence $T_1,T_2,\ldots$ of types is Cauchy if for
every $\epsilon > 0$ there exists $n\in\mathbb N$ such that for
all $i,j\ge n$, the distance between $T_i$~and $T_j$ is smaller
than $\epsilon$.

\begin{theorem}
No matter which rank function is chosen, the metric space of
types is complete in the sense that every Cauchy sequence of
types converges to a type.
\end{theorem}

\begin{proof}
There are two proof goals.
\begin{enumerate}
\item Every Cauchy sequence of types converges.
\item The limit of a Cauchy sequence of types is a type.
\end{enumerate}

Part (1). Let $T_1,T_2,\ldots$ be a Cauchy sequence of types.
Knowing the standard result $\liminf T_n\subseteq\limsup T_n$,
we need only show $\limsup T_n\subseteq\liminf T_n$.

\def\Cauchy{\texttt{cauchy}}
\def\Hasv{{\texttt{has\textunderscore}v}}

Choose an arbitrary value
\[
v\in \limsup T_n = \bigcap_{n=1}^\infty\bigcup_{i = n}^\infty T_i.
\]
Let $\epsilon=2^{-\Rank(v)}>0$. There exists a natural number
$\Cauchy$ such that for all $i,j\ge\Cauchy$, we have
$d(T_i,T_j)<\epsilon$. Since $v$ is a member of the limit
superior, there exists a natural number $\Hasv\ge\Cauchy$ such
that $v\in T_\Hasv$. If we pick any $i\ge\Hasv$, then we have
$v\in T_i$, because otherwise the contradiction
\[
d(T_\Hasv, T_i)\ge2^{-\Rank(v)}=\epsilon
\]
would arise. Thus
\[
v\in
\bigcap_{i=\Hasv}^\infty T_i\subseteq
\bigcup_{n=1}^\infty\bigcap_{i = n}^\infty T_i.
=\liminf T_n.
\]

Part (2). We will verify that the limit
\[
T=\limsup T_n=\liminf T_n
\]
is a nonempty proper subset of $V$ closed downward and closed
under directed supremum.
\end{proof}

\end{document}
