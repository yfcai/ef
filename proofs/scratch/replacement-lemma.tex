\subsection{Contextual extension of constrained typing
judgements}

Let $t$ be a term. The concept of \emph{contextual extension}
helps us match each subterm $s$ of $t$ with the part of of any
constrained typing derivation of $t$ contributed by $s$.

Let $C$ be a one-hole term context. Let $s$ be a term whose free
variables are not captured in $C$. Let $D_s$ be a constrained
typing derivation of $s$. A constrained typing derivation of
$C[s]$ is a contextual extension of $D_s$ if it is constructed
from $D_s$ by constrained typing rules in lockstep as $C[s]$ is
constructed on top of $s$.

More formally, a constrained typing derivation $D$ of $C[s]$ is a
contextual extension of the constrained typing derivation $D_s$
of $s$ relative to the one-hole context $C$ if one of the
following holds.
\begin{enumerate}
\item $C$ is the hole and $D$ is a weakening of $D_s$. (A
weakening of a constrained typing derivation is another
derivation that has some unused extra bindings in its typing
context but is otherwise identical.)
\item The top-level term constructor of $C$ is an abstraction,
application or $n$-nary constant. Let $C_0$ be the immediate
one-hole subcontext of $C$. The derivation $D$ is obtained by one
constrained typing rule from a contextual extension $D_0$ of
$D_s$ relative to $C_0$. If $C_0$ is the $i$th child of $C$, then
$D_0$ is the $i$th structural-recursive antecedent of $D$.
\end{enumerate}


\subsection{Replacement lemma}
\label{sublem}

We set out to prove that if a term can be typed $\tau$, then it
can still be typed $\tau$ after replacing one subterm by
something of a more general type. The substitution lemma common
in preservation proofs is a corollary of the replacement lemma
phrased in this way.

\def\Exi{\emptyset}

Let $p$, $q$ be terms. Let $C$ be a one-hole context capturing no
free variable in $p$ or $q$. Suppose we can derive
\[
\Gamma_c\vdash C[p]:\sigma_{cp}\Given K_{cp}
\]
with a contextual extension of a known derivation of
\[
\Gamma\vdash p:\sigma_p \Given K_p.
\]
(It follows from the definition of contextual extension that
$\Gamma\subseteq\Gamma_c$.) Suppose, under the aforementioned
$\Gamma$, we can derive
\begin{equation}
\Gamma\vdash q:\sigma_q \Given K_q.
\label{eq:subct}
\end{equation}
If $K\supseteq K_{cp}\cup K_p\cup K_q$ is a set of constraints
such that there is a subtype derivation for
\begin{align}
A;\Exi&\vdash
\sigma_{cp}\Sub\tau,
\sigma_q\Sub\sigma_p,
K,
\label{eq:subass}
\end{align}
then there exists $\sigma_{cq}$ and $K_{cq}$ such that the
constrained typing judgement
\[
\Gamma_c \vdash C[q]: \sigma_{cq}\Given K_{cq}
\]
is derivable via a contextual extension of \eqref{eq:subct}.
Moreover, there exists $K'\supseteq K\cup K_{cq}$ such that
\[
A';\Exi \vdash \sigma_{cq}\Sub\tau,K'
\]
is derivable for some $A'$.

\begin{proof}
By induction on the structure of $C$.

\Case$C=\bullet$. It follows that $C[p]=p$ and $C[q]=q$. Choose
$\sigma_{cq}=\sigma_q$ and $K_{cq}=K_q$. The presumed derivation
of $(\Gamma\vdash q:\sigma_q\Given K_q)$ can be weakend to the
desired contextual extension of it. Since the constrained typing
derivation of $C[p]$ is a contextual extension of the one of $p$,
we have $\sigma_{cp}=\sigma_p$ and $K_{cp}=K_p$.
By~\eqref{eq:subass}, we can derive
\[
A;\Exi\vdash \sigma_p\Sub\tau,\sigma_q\Sub\sigma_p,K.
\]
\textsc{S-Trans} produces
\[
A;\Exi\vdash \sigma_q\Sub\tau,K,
\]
which is the desired subtype derivation with $K'=K$.

\Case$C=\Abs y C_0$. Let this be the last step in the presumed
contextual extension of $(\Gamma\vdash p:\sigma_p\Given K_p)$:
\infrule[CT-Abs]
{
\{\alpha_1,\ldots,\alpha_n\}=
\FTV(\sigma_y,\sigma_0,K)-\FTV(\Gamma_c)
\andalso
\Gamma,y:\sigma_y \vdash C_0[p] : \sigma_0 \Given K_0
}
{\Gamma_c\vdash(\Abs y C_0[p]):
(\All{\alpha_1\cdots\alpha_n}\sigma_y\R\sigma_0 \Given K_0)\Given
K_0
}%
By the definition of contextual extension, the antecedent
\[
\Gamma,y:\sigma_y \vdash C_0[p] : \sigma_0 \Given K_0
\]
can be derived via a contextual extension of the presumed
derivation of
\[
\Gamma\vdash p:\sigma_p\Given K_p.
\]
By assumption, there is a subtype derivation for
\[
A;\Exi\vdash
(\All{\alpha_1\cdots\alpha_n}\sigma_y\R\sigma_0 \Given K_0)
\Sub\tau,
\sigma_q\Sub\sigma_p,K
\]
for some $K\supseteq K_0\cup K_p\cup K_q$. Choose fresh type
variables $\beta_1,\ldots,\beta_n$. Some uses of \textsc{S-LE}
followed by \textsc{S-KE} produce
\[
A;\Exi\vdash
\]
\end{proof}
