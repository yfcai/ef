\def\Xdef{w}
\def\Exigent{\emptyset}
We set out to prove that if $t$ can be typed $\tau_0$, then after
a well-typed substitution it can still be typed $\tau_0$.

Let $\Piggy$ be a set of ``piggy-backed'' constraints. Suppose
the following judgements are derivable.
\begin{align}
\Gamma,x:\sigma_x&\vdash t:\tau\Given C_t \notag
\\
\Gamma&\vdash \Xdef:\sigma_\Xdef \Given C_\Xdef \notag
\\
A;\Exigent&\vdash
\sigma_\Xdef\Sub\sigma_x, \tau\Sub\tau_0,C_t\cup C_\Xdef\cup\Piggy
\label{eq:subass}
\end{align}
Then there exist $A'$, $\tau'$, $C'$ and a set of constraints
$\Piggy'\supseteq\Piggy$ such that we can derive
\begin{align*}
A';\Gamma&\vdash t[x\mapsto \Xdef] : \tau'\Given C',\\
A';\Exigent &\vdash \tau'\Sub\tau_0, C_t\cup C_\Xdef\cup C'\cup\Piggy'.
\end{align*}

\begin{proof}
By induction on a constrained typing derivation of $t$ and by
case analysis on the last rule used.

\Case\textsc{CT-Var}: We know $t=x$. The assumptions are
specialized to the simple form below.
\begin{samepage}
\begin{align}
\Gamma,x:\sigma&\vdash x:\sigma_x\Given\Exigent
\notag
\\
\Gamma&\vdash \Xdef:\sigma_\Xdef\Given C_\Xdef
\label{eq:subvar1}
\\
A;\Exigent&\vdash \sigma_\Xdef\Sub\sigma_x,\sigma_x\Sub\tau_0,C_\Xdef\cup\Piggy
\label{eq:subvar2}
\end{align}
\end{samepage}%
Since $t[x\mapsto \Xdef]=\Xdef$, the desired constrained typing judgement
is exactly \eqref{eq:subvar1}. The desired subtype judgement
follows from one use of \textsc{S-Trans} on \eqref{eq:subvar2}.
\[
A;\Exigent\vdash \sigma_\Xdef\Sub\tau_0,C_\Xdef\cup\Piggy.
\]

\Case\textsc{CT-Abs}: $t=\Abs{y:\sigma_y}s$.
\def\yasump{\Gamma,y:\sigma_y\vdash s:\tau_s\Given C_s}
\infrule[CT-Abs]
{\yasump}
{\Gamma\vdash(\Abs{y:\sigma_y}s):\sigma_y\R\tau_s \Given C_s}
Assumption~\eqref{eq:subass} translates to
\[
A;\Exigent\vdash
\sigma_\Xdef\Sub\sigma_x,
\sigma_y\R\tau_s\Sub\tau_0,C_s\cup C_\Xdef\cup\Piggy.
\]
By \textsc{S-Refl},
\[
A;\Exigent\vdash
    \tau_s\Sub\tau_s,
\sigma_\Xdef\Sub\sigma_x,
\sigma_y\R\tau_s\Sub\tau_0,C_s\cup C_\Xdef\cup\Piggy.
\]
Since the constrained typing of $t$ contains a derivation for
\[\yasump,\]
we invoke the induction hypothesis with the piggyback
$(\sigma_y\R\tau_s\Sub\tau_0,\Piggy)$ to conclude that for some
appropriate $A'$ and argument type annotations,
\begin{align}
s'&=s[x\mapsto \Xdef]\notag
\\
A';\Gamma,y:\sigma_y&\vdash s' : \tau_s'\Given C'
\label{eq:subabs1}
\\
A';\Exigent&\vdash
    \tau_s'\Sub\tau_s,
\sigma_y\R\tau_s\Sub\tau_0,C_s\cup C_\Xdef\cup C'\cup\Piggy',
\label{eq:subabs2}
\end{align}
where $\Piggy'\supseteq\Piggy$. By \eqref{eq:subabs1} and
\textsc{CT-Abs},
\[
A';\Gamma\vdash(\Abs{y:\sigma_y}.s)[x\mapsto \Xdef] : \sigma_y\R\tau_s'\Given C'.
\]
By \eqref{eq:subabs2} and \textsc{S-Refl},
\[
A';\Exigent\vdash
    \sigma_y\Sub\sigma_y,
    \tau_s'\Sub\tau_s,
\sigma_y\R\tau_s\Sub\tau_0,C_s\cup C_\Xdef\cup C'\cup\Piggy'.
\]
\textsc{S-Arrow} produces
\[
A';\Exigent\vdash
    \sigma_y\R\tau_s'\Sub\sigma_y\R\tau_s,
\sigma_y\R\tau_s\Sub\tau_0,C_s\cup C_\Xdef\cup C'\cup\Piggy'
\]
and \textsc{S-Trans} concludes
\[
A';\Exigent\vdash
    \sigma_y\R\tau_s'\Sub\tau_0,
C_s\cup C_\Xdef\cup C'\cup\Piggy'.
\]

\Case\textsc{CT-App}: $t=s_1~s_2$.
\infrule[CT-App]
{
\Gamma\vdash s_1:\sigma_1\Given C_1
\andalso
\Gamma\vdash s_2:\sigma_2\Given C_2
}
{\Gamma\vdash s_1~s_2:\beta\Given
\sigma_1\Sub\alpha\R\beta,\sigma_2\Sub\alpha,C_1\cup C_2
}
Set
\[
C_t=\sigma_1\Sub\alpha\R\beta,\sigma_2\Sub\alpha,C_1\cup C_2,
\]
then assumption~\eqref{eq:subass} becomes
\[
A;\Exigent\vdash
  \sigma_\Xdef\Sub\sigma_x,
  \beta\Sub\tau_0,
  C_t\cup C_\Xdef\cup\Piggy.
\]
Set
\[
D=
\sigma_1\Sub\alpha\R\beta,
\sigma_1\Sub\alpha\R\beta,
\sigma_2\Sub\alpha,
\sigma_2\Sub\alpha,C_1\cup C_2\cup C_\Xdef.
\]
Some uses of \textsc{S-Dupe} and \textsc{S-Refl} produce
\[
A;\Exigent\vdash
  \sigma_1\Sub\sigma_1,
  \sigma_\Xdef\Sub\sigma_x,\sigma_\Xdef\Sub\sigma_x,
  \beta\Sub\tau_0,
  D\cup\Piggy.
\]
Invoking the induction hypothesis on $s_1$, we obtain
\begin{align}
s_1'&=s_1[x\mapsto \Xdef]\notag,
\\
\Gamma&\vdash s_1' : \sigma_1' \Given C_1',
\label{eq:subapp1}
\\
A_1';\Exigent&\vdash
  \sigma_\Xdef\Sub\sigma_x,
  \sigma_1'\Sub\sigma_1,
  \beta\Sub\tau_0,
  D\cup\Piggy_1,
\label{eq:subapp2}
\end{align}
where $\Piggy_1\supseteq\Piggy$. From \eqref{eq:subapp2},
\textsc{S-Refl} produces
\begin{align}
A_1';\Exigent\vdash
  \sigma_2\Sub\sigma_2,
  \sigma_\Xdef\Sub\sigma_x,
  \sigma_1'\Sub\sigma_1,
  \beta\Sub\tau_0,
  D\cup C_1'\cup\Piggy_1.
\label{eq:subappf}
\end{align}
Invoking the induction hypothesis on $s_2$ and \eqref{eq:subapp2}
produces
\begin{align}
A_2';\Exigent&\vdash
  \sigma_1'\Sub\sigma_1,
  \sigma_2'\Sub\sigma_2,
  \beta\Sub\tau_0,
  D\cup C_1'\cup C_2'\cup\Piggy_2
\notag\\
&\vdash
  \beta\Sub\tau_0,
  \sigma_1'\Sub\sigma_1,
  \sigma_2'\Sub\sigma_2,
  % dfn of C_t
  \sigma_1\Sub\alpha\R\beta,\sigma_2\Sub\alpha,
  C_t\cup C_\Xdef\cup C_1'\cup C_2'\cup\Piggy_2,
\notag
\end{align}
the last step by expanding $D$ partially to
$\sigma_1\Sub\alpha\R\beta,\sigma_2\Sub\alpha,C_t\cup C_\Xdef$.
Two uses of \textsc{S-Trans} gives us
\begin{align}
A_2';\Exigent\vdash
  \beta\Sub\tau_0,
  \sigma_1'\Sub\alpha\R\beta,\sigma_2'\Sub\alpha,
  C_t\cup C_\Xdef\cup C_1'\cup C_2'\cup\Piggy_2.
\label{eq:subapps}
\end{align}
Use existing type variables $\alpha$, $\beta$ to construct the
desired constrained typing.
\begin{align}
t'&=s_1[x\mapsto \Xdef]~s_2[x\mapsto \Xdef]
\notag\\
\Gamma&\vdash t':
\beta\Given
\sigma_1'\Sub\alpha\R\beta,\sigma_2'\Sub\alpha,
C_1'\cup C_2'
\notag
\end{align}
The desired subtype judgement is derived already: it is
\eqref{eq:subapps}.

\Case\textsc{CT-Con}: The argument here is a straightforward
generalization of \hbox{\textsc{CT-App}} to $n$ subterms and
arbitrary constraints on the types of subterms.

We know $t=c~s_1~\cdots~s_n$ and $\tau=\tau_c$.
%
\infrule[CT-Con]
{
\Gamma\vdash s_1:\sigma_1\Given C_1
\andalso
\cdots
\andalso
\Gamma\vdash s_n:\sigma_n\Given C_n
}
{\Gamma\vdash c~s_1~\cdots~s_n:\tau_c\Given
\sigma_1\Sub\tau_1,\ldots,\sigma_n\Sub\tau_n,
{\textstyle\bigcup_{i=1}^n C_i}
}
%
Write
\[
C_t=\sigma_1\Sub\tau_1,\ldots,\sigma_n\Sub\tau_n,
{\textstyle\bigcup_{i=1}^n C_i}.
\]
Suppose
\begin{align*}
\Gamma&\vdash s_1[x\mapsto \Xdef]:\sigma_1'\Given C_1',\\
&~\vdots\\
\Gamma&\vdash s_n[x\mapsto \Xdef]:\sigma_n'\Given C_n',\\
\Gamma&\vdash t[x\mapsto \Xdef]:\tau_c\Given
\sigma_1'\Sub\tau_1,\ldots,\sigma_n'\Sub\tau_n,
{\textstyle\bigcup_{i=1}^n C_i'}.
\end{align*}
If we write as usual $t'=t[x\mapsto \Xdef]$, then the constraints
generated in the typing of $t'$ are
\[
C'=\sigma_1'\Sub\tau_1,\ldots,\sigma_n'\Sub\tau_n,
{\textstyle\bigcup_{i=1}^n C_i'}.
\]
By assumption, we can derive
\[
A;\Exigent\vdash \sigma_x\Sub\sigma_\Xdef,\tau_c\Sub\tau_0,
\sigma_1\Sub\tau_1,\ldots,\sigma_n\Sub\tau_n,
({\textstyle\bigcup_{i=1}^n C_i})
\cup C_\Xdef\cup\Piggy.
\]
As in previous cases, some judicious use of \textsc{S-Refl} and
\textsc{S-Dupe} allows us to derive
\[
A;\Exigent\vdash \sigma_x\Sub\sigma_\Xdef,\tau_c\Sub\tau_0,
\sigma_1\Sub\sigma_1,
\sigma_1\Sub\tau_1,\ldots,\sigma_n\Sub\tau_n,
C_t\cup C_\Xdef\cup\Piggy.
\]
The induction hypothesis produces
\[
A_1';\Exigent\vdash \sigma_x\Sub\sigma_\Xdef,\tau_c\Sub\tau_0,
\sigma_1'\Sub\sigma_1,
\sigma_1\Sub\tau_1,\ldots,\sigma_n\Sub\tau_n,
C_1\cup C_t\cup C_\Xdef\cup\Piggy_1,
\]
where $\Piggy_1\supseteq\Piggy$, and \textsc{S-Trans} gives us
\[
A_1';\Exigent\vdash \sigma_x\Sub\sigma_\Xdef,\tau_c\Sub\tau_0,
\sigma_1'\Sub\tau_1,
\sigma_2\Sub\tau_2,\ldots,\sigma_n\Sub\tau_n,
C_1'\cup
C_t\cup C_\Xdef\cup
\Piggy_1.
\]
The same trick will produce
\begin{align*}
A_2';\Exigent\vdash\;& \sigma_x\Sub\sigma_\Xdef,\tau_c\Sub\tau_0,
\sigma_1'\Sub\tau_1,
\sigma_2'\Sub\tau_2,\\
&\sigma_3\Sub\tau_3,
\ldots,\sigma_n\Sub\tau_n,
C_1'\cup C_2'\cup
C_t\cup C_\Xdef\cup\Piggy_2
\end{align*}
all the way up to
\begin{align}
A_n';\Exigent&\vdash \tau_c\Sub\tau_0,
\sigma_1'\Sub\tau_1,\ldots,\sigma_n'\Sub\tau_n,
({\textstyle\bigcup_{i=1}^nC_i'})\cup
C_t\cup C_\Xdef\cup\Piggy_n,
\label{eq:subcon}
\end{align}
refraining from duplicating $\sigma_x\Sub\sigma_\Xdef$ at the last
step. If we collect constraints of $C'$ in \eqref{eq:subcon},
then it becomes the desired subtype judgement
\[
A_n';\Exigent\vdash \tau_c\Sub\tau_0,
C_t\cup C_\Xdef\cup C'\cup\Piggy_n.
\]

\Case\textsc{CT-Poly}: Let $\tau$ be the function type produced
in the constrained typing derivation of $t$.
\infrule[CT-Poly]
{\alpha_1,\ldots,\alpha_n\notin\FTV(\Gamma)\cup\FTV(\sigma_x)
\andalso
\Gamma,x:\sigma_x\vdash t:\tau_\alpha\Given C_\alpha
}
{\Gamma,x:\sigma_x\vdash t:
(\All{\alpha_1\cdots\alpha_n}\tau_\alpha\Given C_\alpha)
\Given C_\alpha}
%
By assumption, we can derive
\begin{align}
A;\Exigent &\vdash \sigma_w\Sub\sigma_x,
(\All{\alpha_1\cdots\alpha_n}\tau_\alpha\Given
C_\alpha)\Sub\tau_0,
C_\alpha\cup C_w\cup\Piggy,
\label{eq:subpoly1}
\end{align}
Choose $\beta_1,\ldots,\beta_n$ outside every type variable
occurring in $\Gamma$ or \eqref{eq:subpoly1}, bound or free or
otherwise. Since all $\alpha_i$ and all $\beta_j$ are disjoint
from free type variables in $\Gamma$ and $\sigma_x$, we can
perform $\alpha$-renaming on the derivation of
\[
\Gamma,x:\sigma_x\vdash t : \tau_\alpha\Given C_\alpha
\]
to obtain a derivation of
\def\tBeta{\Gamma,x:\sigma_x\vdash t : \tau_\beta\Given C_\beta}
\[
\tBeta.
\]
Judgement~\eqref{eq:subpoly1} is $\alpha$-equivalent to
\[
A;\Exigent \vdash \sigma_w\Sub\sigma_x,
(\All{\beta_1\cdots\beta_n}\tau_\beta\Given
C_\beta)\Sub\tau_0,
C_\alpha\cup C_w\cup\Piggy.
\]
Several uses of \textsc{S-LE} followed by \textsc{S-CE} gives us
\[
\beta_1,\ldots,\beta_n,A;\Exigent \vdash \sigma_w\Sub\sigma_x,
\tau_\beta\Sub\tau_0,
C_\alpha\cup C_\beta\cup C_w\cup\Piggy.
\]
Calling the induction hypothesis on $(\tBeta)$ produces
\begin{align}
t&'=t[x\mapsto\Xdef],\notag\\
\Gamma&\vdash t':\tau'\Given C',
\label{eq:subpoly2}
\\
A';\Exigent&\vdash
\tau'\Sub\tau_0,
C_\alpha\cup C_w\cup C'\cup(C_\beta\cup\Piggy'),
\label{eq:subpoly3}
\end{align}
which is exactly what we want, with a bigger piggyback.
\end{proof}
