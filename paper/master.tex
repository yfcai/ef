\documentclass{amsart}
\usepackage[foot]{amsaddr}
\usepackage{bcprules,url}

% TODO: reformat bio-data as PLDI requires
\title
{Impredicative Polymorphism\break
for Shakespearean Monkeys}
\author{Yufei Cai}
\address{{\rm
Philipps-Universit\"at Marburg\\
\url{cai@mathematik.uni-marburg.de}\\
05D06 Mehrzweckgeb\"aude, Hans-Meerwein Str., 35039 Marburg, Germany}}

\let\phi=\varphi % default phi looks like empty set
\allowdisplaybreaks
\swapnumbers
\newtheorem{theorem}[subsection]{Theorem}
\newtheorem{lemma}[subsection]{Lemma}
\newtheorem{corollary}[subsection]{Corollary}

\def\MLF!{ML$^\text{F}$}

\usepackage{stmaryrd}
\usepackage[cmtip,all]{xy}
\newcommand{\nc}{\newcommand}
\newcommand{\DotExpr}[2]{#1 #2.\ }
\newenvironment{syntax}{\[\begin{array}{rclr}}{\end{array}\]}

\makeatletter
\renewcommand{\@secnumfont}{\bfseries}
\makeatother
\def\thesubsection{\arabic{section}.\arabic{subsection}}
\def\+={\text{+=}}
\setlength\fboxsep{0pt}

\nc\Abs       {\DotExpr\lambda}
\nc\Add       {\mathrm{add}}
\nc\All       {\DotExpr\forall}
\nc\Allin     {\forall\mathrm{in}}
\nc\Allex     {\forall\mathrm{ex}}
\nc\AbortCC   {\mathrm{abort/cc}}
\nc\ApplyCC   {\mathrm{apply/cc}}
\nc\B         {\mathbb}
\nc\Bool      {\mathrm{Bool}}
\nc\Bot       {\texttt{Bot}}
\nc\BoxIt[1]  {\framebox{\raisebox{0.1ex}[1.5ex][0.2ex]{$#1$}}}
\nc\Brouwer   {\operatorname{Brouwer}}
\nc\Case      {\medbreak\noindent\textbf{Case}~}
\nc\Cap       {\MakeL\cap}
\nc\Closure   {\mathrm{closure}}
\nc\Cps       {\operatorname{cps}}
\nc\Cup       {\MakeL\cup}
\nc\Dn[1]     {(#1 \R \bot) \R \bot}
\nc\Dni       {\mathrm{\neg\neg I}}
\nc\Dom       {\mathrm{dom}}
\nc\Down      {\mathord\downarrow}
\nc\Eco       {\mathit{Eco}}
\nc\Env       {\mathit{Env}}
\nc\env       {\mathit{env}}
\nc\Erase     {\mathrm{erase}}
\nc\Ex        {\DotExpr\exists}
\nc\Exin      {\exists\mathrm{in}}
\nc\Exex      {\exists\mathrm{ex}}
\nc\Exmid     {\text{excluded-middle}}
\nc\False     {\mathrm{false}}
\nc\Fix       {\mathrm{fix}}
\nc\FTV       {\operatorname{FTV}}
\nc\FV        {\operatorname{FV}}
\nc\Given     {\text{ given }}
\nc\Halt      {\mathrm{halt}}
\nc\Id        {\mathit{id}}
\nc\If        {\mathrm{if}}
\nc\Ideals    {\mathcal{I}}
\nc\Indent    {\hspace{3em}}
\nc\Int       {\texttt{Int}}
\nc\Inject    {\mathrm{inject}}
\nc\JCT       {J_{\rm CT}}
\nc\JF        {J_{\rm F}}
\nc\JS        {J_{\rm S}}
\nc\LHS       {\operatorname{LHS}}
\nc\Mgsr      {\operatorname{mgsr}}
\nc\MakeT[1]  {\ \BoxIt{#1}\ }
\nc\MakeL[1]  {\ \mathrm{:\hspace{-2pt}#1\hspace{-2pt}:}\ }
\nc\Or        {\ | \ }
\nc\Pack      {\mathrm{pack}}
\nc\Par       {\subsection}
\nc\Piggy     {\mathrm{Piggy}}
\nc\Plus      {\MakeT+}
\nc\Prune     {\operatorname{prune}}
\nc\Rank      {\mathrm{rank}}
\nc\Recall    {\DotExpr\Xi} % looks like ∃
\nc\Redo      {\operatorname{redo}}
\nc\RHS       {\operatorname{RHS}}
\nc\Sem[1]    {\SemSlim{~#1~}}
\nc\SemSlim[1]{\llbracket#1\rrbracket}
\nc\String    {\texttt{String}}
\nc\Sub       {\sqsubseteq}
\nc\Sup       {\sqsupseteq}
\nc\Tabs      {\DotExpr\Lambda}
\nc\Tech      {\text{technical}}
\nc\Times     {\MakeT\times}
\nc\True      {\mathrm{true}}
\nc\Type      {\mathbf{Type}}
\nc\R         {\rightarrow}
\nc\RL        {\MakeL{\BoxIt\R}}
\nc\RT        {\MakeT\R}
%\nc\Red       {\xymatrix{{}\ar@{~>}[r]&{}}}
\nc\Red       {\rightsquigarrow}
\nc\RR        {\Rightarrow}
\nc\Top       {\texttt{Top}}
\nc\Unpack    {\mathrm{unpack}}
\nc\Undo      {\operatorname{undo}}
\nc\Unify     {\operatorname{unify}}
\nc\Up        {\mathord{\uparrow}}
\nc\Update[1] {#1\Updated}
\nc\Updated   {\text{ updated }}
\nc\VST       {\vdash_{\textrm{ST}}}
\nc\VSub      {\vdash_{\textrm S}}
\nc\Wrong     {\textsc{wrong}}

\begin{document}
\maketitle

\subsection{Problem statement}

The problem is, first-class polymorphism has hardly any presence
in everyday computing.

First-class polymorphism is a wonderful feature for statically
typed functional languages. It refers to the ability to not only
define polymorphic functions, but also take them as arguments and
use them polymorphically in the body of any definition. With
first-class polymorphism, a library function can sort arrays of
different things with a user-defined sorting algorithm. Less
mundane applications include B\"ohm-Berarducci encoding of
algebraic data types~\cite{Boehm85}, scrap-your-boilerplate
pattern~\cite{Laemmel03}, foldable higher-order abstract
syntax~\cite{Washburn03}, and ST monad~\cite{Launchbury94}. Those
applications were discovered despite this age's far-from-perfect
support for first-class polymorphism. One may wonder how much
more would be found expressible if first-class polymorphism were
effortless to use.

A type system with first-class polymorphism would become
practical sooner if it were easy to use and easy to implement. A
perspicuous type safety proof could help as well. Such a type
system is my goal.

TODO: short summary of what is meant by ``ease of use'' and
``ease of implementation''.

\subsection{Historical background}


\begin{thebibliography}{99}
%\bibitem{Hamkins10}
%Joel David Hamkins
%(\hbox{\url{http://mathoverflow.net/users/1946/joel-david-hamkins}}).\\
%Collection of subsets closed under union and intersection.\\
%URL (version: 2010-01-11):
%\hbox{\url{http://mathoverflow.net/q/11451}}

\bibitem{Boehm85}
Corrado B\"ohm and Alessandro Berarducci.
Automatic synthesis of typed $\Lambda$-programs on term algebras.
\emph{Theoretic Computer Science} 39 (1985).

\bibitem{Laemmel03}
Ralf L\"ammel and Simon Peyton Jones.
Scrap your boilerplate: a practical approach to generic
programming.
\emph{Proceedings of ACM SIGPLAN Workshop on Types in Language
Design and Implementation} (TLDI 2003).

\bibitem{Launchbury94}
John Launchbury and SImon Peyton Jones.
Lazy functional state threads.
\emph{Proceedings of the ACM SIGPLAN 1994 Conference on
Programming Language Design and Implementation} (PLDI 1994).

\bibitem{Milner78}
Robin Milner.
A theory of type polymorphism in programming.
\emph{Journal of Computer and System Sciences} 17 (1978).

\bibitem{Washburn03}
Geoffrey Washburn and Stephanie Weirich.
Boxes go bananas: encoding higher-order abstract syntax with
parametric polymorphism.
\emph{Proceedings of the Eighth ACM SIGPLAN International
Conference on Functional Programming} (ICFP 2003).

\bibitem{Wright94}
Andrew K. Wright and Matthias Felleisen.
A syntactic approach to type soundness.
\emph{Information and Computation} 115-1 (1994).
\end{thebibliography}
\end{document}
