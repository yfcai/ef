\begin{thebibliography}{99}
%\bibitem{Hamkins10}
%Joel David Hamkins
%(\hbox{\url{http://mathoverflow.net/users/1946/joel-david-hamkins}}).\\
%Collection of subsets closed under union and intersection.\\
%URL (version: 2010-01-11):
%\hbox{\url{http://mathoverflow.net/q/11451}}

\bibitem{Boehm85}
Corrado B\"ohm and Alessandro Berarducci.
Automatic synthesis of typed $\Lambda$-programs on term algebras.
\emph{Theoretic Computer Science} 39 (1985).

\bibitem{Jones07}
Simon Peyton Jones, Dimitrios Vytiniotis, Stephanie Weirich, and
Mark Shields.
Practical type inference for arbitrary-rank types.
\emph{Journal of Functional Programming} 17(1), 2007.

\bibitem{Kiselyov13}
Oleg Kiselyov, Amr Sabry, and Cameron Swords.
Extensible effects: an alternative to monad transformers.
\emph{Proceedings of the 2013 ACM SIGPLAN Symposium on Haskell}
(Haskell 2013).

\bibitem{Laemmel03}
Ralf L\"ammel and Simon Peyton Jones.
Scrap your boilerplate: a practical approach to generic
programming.
\emph{Proceedings of ACM SIGPLAN Workshop on Types in Language
Design and Implementation} (TLDI 2003).

\bibitem{Launchbury94}
John Launchbury and Simon Peyton Jones.
Lazy functional state threads.
\emph{Proceedings of the ACM SIGPLAN 1994 Conference on
Programming Language Design and Implementation} (PLDI 1994).

\bibitem{LeBotlan03}
Didier Le Botlan and Didier R\'emy.
\MLF!: Raising ML to the power of System~F.
\emph{Proceedings of the Eighth ACM SIGPLAN International
Conference on Functional Programming} (ICFP~2003).

\bibitem{Leijen09}
Daan Leijen.
Flexible types: robust type inference for first-class
polymorphism.
\emph{Proceedings of the 36th Annual ACM SIGPLAN-SIGACT Symposium
on Principles of Programming Languages} (POPL~2009).

\bibitem{Marlow10}
Simon Marlow (editor).
Haskell 2010 language report.
URL (version: 08 March 2014):\\
\hbox{\url{http://www.haskell.org/onlinereport/haskell2010/}}

\bibitem{Milner78}
Robin Milner.
A theory of type polymorphism in programming.
\emph{Journal of Computer and System Sciences} 17 (1978).

\bibitem{Milner97}
Robin Milner, Robert Harper, David MacQueen, and Mads Tofte.
\emph{The Definition of Standard ML (Revised)}.
MIT Press, 1997. ISBN~0-262-63181-4.

\bibitem{RemyMLF}
Didier R\'emy, Didier Le Botlan, Boris Yakobowski, Paolo Herms,
Gabriel Scherer.
\MLF!.\\
URL (version: 08 March 2014):
\hbox{\url{http://gallium.inria.fr/~remy/mlf/}}

\bibitem{Reynolds74}
John C. Reynolds.
Towards a theory of type structure.
\emph{Programming Symposium: Proceedings, Colloque sur
la Programmation}. Springer-Verlag, 1974. Lecture Notes in
Computer Science 19.

\bibitem{Schubert98}
Alesky Schubert.
Second-order unification and type inference for Church-style
polymorphism.
\emph{Proceedings of the 25th ACM SIGPLAN-SIGACT Symposium on
Principles of Programming Languages} (POPL 1998).

\bibitem{Vytiniotis06}
Dimitrios Vytiniotis, Stephanie Weirich, and Simon Peyton Jones.
Boxy types: type inference for higher rank and impredicativity.
\emph{Proceedings of the Eleventh ACM SIGPLAN International
Conference on Functional Programming} (ICFP 2006)

\bibitem{Washburn03}
Geoffrey Washburn and Stephanie Weirich.
Boxes go bananas: encoding higher-order abstract syntax with
parametric polymorphism.
\emph{Proceedings of the Eighth ACM SIGPLAN International
Conference on Functional Programming} (ICFP 2003).

\bibitem{Wells96}
J. B. Wells.
Typability and type checking in the second-order
$\lambda$-calculus are equivalent and undecidable.
\emph{Proceedings of the Ninth Annual IEEE Symposium on Logic in
Computer Science} (LICS 1996).

\bibitem{Wright94}
Andrew K. Wright and Matthias Felleisen.
A syntactic approach to type soundness.
\emph{Information and Computation} 115-1 (1994).
\end{thebibliography}
