% Template file for an a0 landscape poster.
% Written by Graeme, 2001-03 based on Norman's original microlensing
% poster.
%
% See discussion and documentation at
% <http://www.astro.gla.ac.uk/users/norman/docs/posters/> 
%
% $Id$


% Default mode is landscape, which is what we want, however dvips and
% a0poster do not quite do the right thing, so we end up with text in
% landscape style (wide and short) down a portrait page (narrow and
% long). Printing this onto the a0 printer chops the right hand edge.
% However, 'psnup' can save the day, reorienting the text so that the
% poster prints lengthways down an a0 portrait bounding box.
%
% 'psnup -w85cm -h119cm -f poster_from_dvips.ps poster_in_landscape.ps'

\documentclass[a0]{a0poster}
% You might find the 'draft' option to a0 poster useful if you have
% lots of graphics, because they can take some time to process and
% display. (\documentclass[a0,draft]{a0poster})

\pagestyle{empty}
\setcounter{secnumdepth}{0}

% The textpos package is necessary to position textblocks at arbitary 
% places on the page.
\usepackage[absolute]{textpos}

% Graphics to include graphics. Times is nice on posters, but you
% might want to switch it off and go for CMR fonts.
\usepackage{graphics,wrapfig,times}

% These colours are tried and tested for titles and headers. Don't
% over use color!
\usepackage{color}
\definecolor{DarkBlue}{rgb}{0.1,0.1,0.5}
\definecolor{Red}{rgb}{0.9,0.0,0.1}

% see documentation for a0poster class for the size options here
\let\Textsize\normalsize
\def\Head#1{\noindent\hbox to \hsize{\hfil{\LARGE\color{DarkBlue} #1}}\bigskip}
\def\LHead#1{\noindent{\LARGE\color{DarkBlue} #1}\bigskip}
\def\Subhead#1{\noindent{\large\color{DarkBlue} #1}\bigskip}
\def\Title#1{\noindent{\VeryHuge\color{Red} #1}}


% Set up the grid
%
% Note that [40mm,40mm] is the margin round the edge of the page --
% it is _not_ the grid size. That is always defined as 
% PAGE_WIDTH/HGRID and PAGE_HEIGHT/VGRID. In this case we use
% 23 x 12. This gives us three columns of width 7 boxes, with a gap of
% width 1 in between them. 12 vertical boxes is a good number to work
% with.
%
% Note however that texblocks can be positioned fractionally as well,
% so really any convenient grid size can be used.
%
\TPGrid[40mm,40mm]{23}{12}      % 3 cols of width 7, plus 2 gaps width 1

\parindent=0pt
\parskip=0.5\baselineskip

\begin{document}

% Understanding textblocks is the key to being able to do a poster in
% LaTeX. In
%
%    \begin{textblock}{wid}(x,y)
%    ...
%    \end{textblock}
%
% the first argument gives the block width in units of the grid
% cells specified above in \TPGrid; the second gives the (x,y)
% position on the grid, with the y axis pointing down.

% You will have to do a lot of previewing to get everything in the 
% right place.

% This gives good title positioning for a portrait poster.
% Watch out for hyphenation in titles - LaTeX will do it
% but it looks awful.
\begin{textblock}{23}(0,0)
\Title{Look at this Poster, which is in Landscape mode}
\end{textblock}

\begin{textblock}{8}(0,1.0)
\LHead{Esther the Fester$^1$, John de Quincy$^1$, Andrew Marx Protestor$^2$
\hfil\break
\textsl{1:Department of Physics and Astronomy, University of
  Glasgow\\2:Department of Physics and Astronomy, The Closed Box}}
\end{textblock}


% Uni logo in the top right corner. A&A in the bottom left. Gives a
% good visual balance, but you may want to change this depending upon
% the graphics that are in your poster.
\begin{textblock}{2}(0,10)
Your logo here
%\includegraphics{/usr/local/share/images/AandA.epsf}
\end{textblock}

\begin{textblock}{2}(21.2,0)
Another logo here
%\resizebox{2\TPHorizModule}{!}{\includegraphics{/usr/local/share/images/GUVIu/GUVIu.eps}}
\end{textblock}


\begin{textblock}{7}(0,2.5)
\hrule\medskip

\Head{Summary}

\slshape

 Blah, blah, blah.

  We three sit in the grid.\\
  Like spades hanging in the wind.\\
  Falling down.

  Concrete poetry is a concrete poetry is a concrete poetry is a 
  concrete poetry is a concrete poetry is a concrete poetry is a 
  concrete poetry is a concrete poetry is a concrete poetry is a 
  concrete poetry is a concrete poetry is a concrete poetry is a 
  concrete poetry is a concrete poetry is a concrete poetry is a 
  concrete poetry is a concrete poetry is a concrete poetry is a 
  concrete poetry is a concrete poetry is a concrete poetry is a 
  CON

  \[
  x' = \frac{1}{x}
  \]

  Circular arguments go round and round, they don't go down the page
  like this, hence this is not a circular arguments go round and round, they don't go down the page
  like this, hence this is not a circular arguments go round and round, they don't go down the page
  like this, hence this is not a circular arguments go round and round, they don't go down the page
  like this, hence this is not a circular arguments go round and round, they don't go down the page
  like this, hence this is not a circular arguments go round and round, they don't go down the page
  like this, hence this is not a circular arguments go round and round, they don't go down the page
  like this, hence this is not a circular arguments go round and round, they don't go down the page
  like this, hence this is not a 

\bigskip
\hrule
\end{textblock}

\begin{textblock}{7}(16,8)

\Head{Conclusions}

Landscape posters are possible (and there was much rejoycing).
Landscape posters are possible (and there was much rejoycing).
Landscape posters are possible (and there was much rejoycing).
Landscape posters are possible (and there was much rejoycing).
Landscape posters are possible (and there was much rejoycing).
  
\end{textblock}


\end{document}
