\documentclass{amsart}
\usepackage[foot]{amsaddr}
\usepackage{bcprules,url}

\title
{Impredicative Polymorphism\break
for Shakespearean Monkeys}
\author{Yufei Cai}
\address{{\rm
Philipps-Universit\"at Marburg\\
\url{cai@mathematik.uni-marburg.de}\\
05D06 Mehrzweckgeb\"aude, Hans-Meerwein Str., 35039 Marburg, Germany}}

\let\phi=\varphi % default phi looks like empty set
\allowdisplaybreaks
\swapnumbers
\newtheorem{theorem}[subsection]{Theorem}
\newtheorem{lemma}[subsection]{Lemma}
\newtheorem{corollary}[subsection]{Corollary}

\def\MLF!{ML$^\text{F}$}

\usepackage{stmaryrd}
\usepackage[cmtip,all]{xy}
\newcommand{\nc}{\newcommand}
\newcommand{\DotExpr}[2]{#1 #2.\ }
\newenvironment{syntax}{\[\begin{array}{rclr}}{\end{array}\]}

\makeatletter
\renewcommand{\@secnumfont}{\bfseries}
\makeatother
\def\thesubsection{\arabic{section}.\arabic{subsection}}
\def\+={\text{+=}}
\setlength\fboxsep{0pt}

\nc\Abs       {\DotExpr\lambda}
\nc\Add       {\mathrm{add}}
\nc\All       {\DotExpr\forall}
\nc\Allin     {\forall\mathrm{in}}
\nc\Allex     {\forall\mathrm{ex}}
\nc\AbortCC   {\mathrm{abort/cc}}
\nc\ApplyCC   {\mathrm{apply/cc}}
\nc\B         {\mathbb}
\nc\Bool      {\mathrm{Bool}}
\nc\Bot       {\texttt{Bot}}
\nc\BoxIt[1]  {\framebox{\raisebox{0.1ex}[1.5ex][0.2ex]{$#1$}}}
\nc\Brouwer   {\operatorname{Brouwer}}
\nc\Case      {\medbreak\noindent\textbf{Case}~}
\nc\Cap       {\MakeL\cap}
\nc\Closure   {\mathrm{closure}}
\nc\Cps       {\operatorname{cps}}
\nc\Cup       {\MakeL\cup}
\nc\Dn[1]     {(#1 \R \bot) \R \bot}
\nc\Dni       {\mathrm{\neg\neg I}}
\nc\Dom       {\mathrm{dom}}
\nc\Down      {\mathord\downarrow}
\nc\Eco       {\mathit{Eco}}
\nc\Env       {\mathit{Env}}
\nc\env       {\mathit{env}}
\nc\Erase     {\mathrm{erase}}
\nc\Ex        {\DotExpr\exists}
\nc\Exin      {\exists\mathrm{in}}
\nc\Exex      {\exists\mathrm{ex}}
\nc\Exmid     {\text{excluded-middle}}
\nc\False     {\mathrm{false}}
\nc\Fix       {\mathrm{fix}}
\nc\FTV       {\operatorname{FTV}}
\nc\FV        {\operatorname{FV}}
\nc\Given     {\text{ given }}
\nc\Halt      {\mathrm{halt}}
\nc\Id        {\mathit{id}}
\nc\If        {\mathrm{if}}
\nc\Ideals    {\mathcal{I}}
\nc\Indent    {\hspace{3em}}
\nc\Int       {\texttt{Int}}
\nc\Inject    {\mathrm{inject}}
\nc\JCT       {J_{\rm CT}}
\nc\JF        {J_{\rm F}}
\nc\JS        {J_{\rm S}}
\nc\LHS       {\operatorname{LHS}}
\nc\Mgsr      {\operatorname{mgsr}}
\nc\MakeT[1]  {\ \BoxIt{#1}\ }
\nc\MakeL[1]  {\ \mathrm{:\hspace{-2pt}#1\hspace{-2pt}:}\ }
\nc\Or        {\ | \ }
\nc\Pack      {\mathrm{pack}}
\nc\Par       {\subsection}
\nc\Piggy     {\mathrm{Piggy}}
\nc\Plus      {\MakeT+}
\nc\Prune     {\operatorname{prune}}
\nc\Rank      {\mathrm{rank}}
\nc\Recall    {\DotExpr\Xi} % looks like ∃
\nc\Redo      {\operatorname{redo}}
\nc\RHS       {\operatorname{RHS}}
\nc\Sem[1]    {\SemSlim{~#1~}}
\nc\SemSlim[1]{\llbracket#1\rrbracket}
\nc\String    {\texttt{String}}
\nc\Sub       {\sqsubseteq}
\nc\Sup       {\sqsupseteq}
\nc\Tabs      {\DotExpr\Lambda}
\nc\Tech      {\text{technical}}
\nc\Times     {\MakeT\times}
\nc\True      {\mathrm{true}}
\nc\Type      {\mathbf{Type}}
\nc\R         {\rightarrow}
\nc\RL        {\MakeL{\BoxIt\R}}
\nc\RT        {\MakeT\R}
%\nc\Red       {\xymatrix{{}\ar@{~>}[r]&{}}}
\nc\Red       {\rightsquigarrow}
\nc\RR        {\Rightarrow}
\nc\Top       {\texttt{Top}}
\nc\Unpack    {\mathrm{unpack}}
\nc\Undo      {\operatorname{undo}}
\nc\Unify     {\operatorname{unify}}
\nc\Up        {\mathord{\uparrow}}
\nc\Update[1] {#1\Updated}
\nc\Updated   {\text{ updated }}
\nc\VST       {\vdash_{\textrm{ST}}}
\nc\VSub      {\vdash_{\textrm S}}
\nc\Wrong     {\textsc{wrong}}

\begin{document}

\def\OK{\\[5pt]}
\begin{tabular}{ll}
Name & Yufei Cai\OK
Affiliation & Philipps-Universit\"at Marburg\OK
Email & \url{cai@mathematik.uni-marburg.de} \OK
Address & 05D06 Mehrzweckgeb\"aude,\\
              & Hans-Meerwein Stra\ss e,\\
              & 35039 Marburg, Germany \OK
Advisor & Prof.\ Dr.\ Klaus Ostermann \OK
ACM number & 9866679 \OK
Category & Graduate student \OK
Research title & A simple impredicative type system
\end{tabular}

\bigbreak

\subsection{Problem statement}

The problem is, first-class polymorphism does not have enough
presence in everyday computing.

First-class polymorphism is a wonderful feature for statically
typed functional languages. It refers to the ability to not only
define polymorphic functions, but also take them as arguments and
use them polymorphically in the body of any definition. With
first-class polymorphism, a library function can sort arrays of
different things with a user-defined sorting algorithm. Less
mundane applications include B\"ohm-Berarducci encoding of
algebraic data types~\cite{Boehm85}, extensible
effects~\cite{Kiselyov13}, scrap-your-boilerplate
pattern~\cite{Laemmel03}, ST monad~\cite{Launchbury94}, and
foldable higher-order abstract syntax~\cite{Washburn03}. Those
applications were discovered despite this age's far-from-perfect
support for first-class polymorphism. One may wonder how much
more would be found expressible if it were effortless to use.

A type system with first-class polymorphism would sooner become
practical if it were easy to use and easy to implement.
Perspicuous safety and utility proofs could help as well. Such a
type system is my goal.

\subsection{Background}

System~F~\cite{Reynolds74} is the first type system to support
first-class polymorphism. In System~F, types of function
arguments are fully annotated. Polymorphism is expressed through
universally quantified types. The user must specialize
polymorphic functions via type arguments before applying them to
other terms. Researchers noticed the heavy notational overhead of
System~F from the very beginning and sought ways of avoiding it,
until type inference was proven
undecidable~\cite{Schubert98,Wells96}. The burden of type
annotation is both quantitative and qualitative. The types
$\All\alpha\All\beta\alpha\R\beta\R\alpha$ and
$\All\beta\All\alpha\alpha\R\beta\R\alpha$ express essentially
identical behavior, but users have to distinguish between them to
put type arguments in the right order. Correct usage of a library
function requires more than keeping its input/output behavior in
mind.

Rank-N-Types~\cite{Jones07} adds first-class polymorphism to
languages similar to Haskell~\cite{Marlow10} or
ML~\cite{Milner97}. At the heart of Rank-N-Types is an algorithm
capable of reconstructing all unquantified type arguments in a
System~F term. It is an intellectual continuation of
Algorithm~W~\cite{Milner78} in the tradition of \emph{predicative
polymorphism}, where the instantiations of type variables are
restricted so as to circumvent undecidability. Rank-N-Types
eliminates a significant portion of notational overhead and is
usable in
practice~\cite{Kiselyov13,Laemmel03,Launchbury94,Washburn03}.
However, use cases tend to require quantified type arguments,
which the user must in some way provide. The programmer's mental
model remains complex. One has to think about type applications
to decide where an annotation may be necessary. Other predicative
systems have the same drawback: They reduce manual labor without
reducing mental effort.

\MLF!~\cite{LeBotlan03,RemyMLF} is an impredicative type system.
It requires annotation only on a subset of polymorphic arguments.
System~F is easily embedded into \MLF!. The theoretic
investigation of related systems is incredibly
thorough~\cite{RemyMLF}. \MLF! has every potential for ease of
use. That it is not a household name yet, we may blame on the
complexity of the underlying theory and of the type inference
algorithm. In \MLF!, a quantified type variable can be subject to
a rigid or flexible constraint against any other type. It can be
hard to discern the input/output behavior of a function given its
type. The 4 years between the initial publication of \MLF! and
the first prototype\footnote{ Boris Yakobowski developed the
first prototype of \MLF! and published it on 8 April
2008~\cite{RemyMLF}. } indicates the difficulty of
implementation. Flexible types~\cite{Leijen09} and Boxy
types~\cite{Vytiniotis06} are attempts to simplify \MLF! or
otherwise shield the programmer from its complexity; they inherit
at least a part of its theoretical and practical hardness.

\subsection{Approach} I believe polymorphism can be understood
\emph{extensionally}, in terms of functions passing information
unaltered from contravariant positions to covariant positions. If
a type system is no more than an encoding of statements about the
behavior of functions, then neither have programmers anything
beyond the behavior of functions to worry about.

\subsection{Contribution and results}

I hope to eventually obtain a type system that is easy to use and
trivial to implement, where each type has a clear, intuitive
meaning. Such a type system will make first-class polymorphism
available for programmers who want to get things done quickly,
with as little distraction as possible.

For now, I have put together a type system according to
extensional intuition. An implementation guide using typing rules
would fit in one page. Anyone should be able to implement it in a
week, so long as they have some experience in unification-based
type inference for simply typed lambda calculus. I proved type
safety via the syntactic approach~\cite{Wright94}, by progress
and preservation, and tested the ease of use in a case study of
B\"ohm-Berarducci encoding of lists. The case study consists of
type signatures in System~F syntax, and term definitions in
lambda calculus with optional argument type annotations. The case
study requires only one annotation in System~F syntax to
typecheck. Type errors due to programmer oversight are caught.

It remains to find a precise extensional meaning for the type
system. In addition to case studies, it would be nice to
\emph{prove} the ease of use. For instance, each System~F term
might be well-typed if we erase type abstractions and type
applications (This is the case for System~F terms of depth at
most 5, with variables at depth 1). Or there might be some way to
determine when an argument needs annotation; most polymorphic
arguments in the case study do not.

\begin{thebibliography}{99}
%\bibitem{Hamkins10}
%Joel David Hamkins
%(\hbox{\url{http://mathoverflow.net/users/1946/joel-david-hamkins}}).\\
%Collection of subsets closed under union and intersection.\\
%URL (version: 2010-01-11):
%\hbox{\url{http://mathoverflow.net/q/11451}}

\bibitem{Boehm85}
Corrado B\"ohm and Alessandro Berarducci.
Automatic synthesis of typed $\Lambda$-programs on term algebras.
\emph{Theoretic Computer Science} 39 (1985).

\bibitem{Laemmel03}
Ralf L\"ammel and Simon Peyton Jones.
Scrap your boilerplate: a practical approach to generic
programming.
\emph{Proceedings of ACM SIGPLAN Workshop on Types in Language
Design and Implementation} (TLDI 2003).

\bibitem{Launchbury94}
John Launchbury and SImon Peyton Jones.
Lazy functional state threads.
\emph{Proceedings of the ACM SIGPLAN 1994 Conference on
Programming Language Design and Implementation} (PLDI 1994).

\bibitem{Milner78}
Robin Milner.
A theory of type polymorphism in programming.
\emph{Journal of Computer and System Sciences} 17 (1978).

\bibitem{Washburn03}
Geoffrey Washburn and Stephanie Weirich.
Boxes go bananas: encoding higher-order abstract syntax with
parametric polymorphism.
\emph{Proceedings of the Eighth ACM SIGPLAN International
Conference on Functional Programming} (ICFP 2003).

\bibitem{Wright94}
Andrew K. Wright and Matthias Felleisen.
A syntactic approach to type soundness.
\emph{Information and Computation} 115-1 (1994).
\end{thebibliography}
\end{document}
